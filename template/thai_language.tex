\usepackage{fontspec}
\usepackage{xunicode}
\usepackage{xltxtra}
%\defaultfontfeatures{Scale=1.23}
%\XeTeXlinebreaklocale “th_TH” % สำหรับตัดคำ
%\setmainfont[Scale=1.23]{THSarabunPSK}


%--------------------------------ภาษาไทย--------------------------------
\defaultfontfeatures{Scale=1.23}
\setmainfont[Scale=1.23]{THSarabunNew}
\XeTeXlinebreaklocale "th_TH"
\XeTeXlinebreakskip = 0pt plus 1pt %----ตัดคำขอบขวาจะตรงr

%==============================================
\renewcommand{\chaptername}{บทที่}
\makeatother
\newtheorem{Theorem}{\bf ทฤษฎีบท}[chapter]
\newtheorem{Definition}{\bf  บทนิยาม}
\newtheorem{Lemma}{ \bf ทฤษฎีบทประกอบ}[section]   
\newtheorem{Proposition}{\bf ประพจน์}[section]
\newtheorem{Remark}{\bf หมายเหตุ}[section]
\newtheorem{Corollary}{\bf บทแทรก}[section]
\newtheorem{Example}{\bf ตัวอย่าง}[section]
\newtheorem{Exercise}{\bf แบบฝึกหัด}[chapter]
\newtheorem{Observation}{\bf ข้อสังเกต}[section]
\renewcommand{\contentsname}{สารบัญ}
\renewcommand{\bibname}{บรรณานุกรม}
\renewcommand{\indexname}{ดัชนี}
\renewcommand\listfigurename{สารบัญรูป}
\renewcommand\listtablename{สารบัญตาราง}
\renewcommand{\appendixname}{ภาคผนวก}

\renewcommand\tablename{ตารางที่}
\renewcommand\figurename{รูปที่}
\renewcommand{\bibname}{บรรณานุกรม}						
\counterwithin{figure}{section}
\numberwithin{equation}{section}		