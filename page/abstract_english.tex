\thispagestyle{empty}
\vspace{2 cm}
{\huge \bf Abstract}\addcontentsline{toc}{chapter}{บทคัดย่อ}

\vspace{2 cm}

\hspace{1cm} The classical total variation (TV) model has made great successes in image inpainting due to the edge-preserving property of the TV regularization. However, it is difficult in developing an efficient numerical method to ensure that numerical solutions satisfy this requirement because of the non-differentiability and non-linearity of the TV regularization. In this work we focus on computational challenges arising in approximately solving TV-based image inpainting model. Motivated by many efficient numerical algorithms in image denoising, we propose to use the so-called split Bregman method (SBM) in this work. At each iteration, the computation of our proposed SBM requires to solve two subproblems. On one hand for the first subproblem, it is difficult to obtain exact solution. On the other hand for the second subproblem, it has a closed-form solution. To this end, we propose a new numerical algorithm (our first algorithm) based on an alternating minimization method in a multi-resolution framework to obtain a fast and accurate numerical solution for TV-based image inpainting model. We further study the problem of removing subtitles from animes. It is found that TV-based image inpainting model can be improved to deliver visually pleasing results. In order to solve the modified model, we propose the skipping and borrowing algorithm (our second algorithm) including the first algorithm to efficiently eliminate subtitles from the animes. Numerical tests on synthetic and real ancient Thai painting images and confirm first that our first algorithm is more computationally efficient than some traditional methods in producing the high quality results. Second, the numerical tests show that the second algorithm is fast in delivering high quality of the restored animes.   

\vspace{1 cm}

{\bf{Keywords:}} Numerical method, Total Variation, Image inpainting \\
\newpage