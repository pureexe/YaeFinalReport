\thispagestyle{empty}
\vspace{2 cm}
{\huge \bf Abstract}\addcontentsline{toc}{chapter}{บทคัดย่อ}

\vspace{2 cm}
\hspace{1cm} This work has presented a new numerical algorithm for TV-based image inpainting with its application for restoring ancient Thai painting images and removing subtitles from animes. This numerical algorithm is used for solving Total Variation model for inpainting that was developed by Chan and Shen \cite{ref:rof-inpaint-chan-shen}. To restoring ancient Thai painting. We will use Splitbregman algorithm which has developed by Goldstien and Osher \cite{ref:splitbergman-inpaint} with image pyramid which have developed by Andelson and His research group \cite{ref:image-pyramid} and reduced iteration number in original pyramid resolution. This algorithm have better quality when measuring with Peak Signal Noise Ratio (PSNR) and 7 times faster than Splitbregman algorithm. For removing subtitle from anime, We have used the algorithm that we have developed for ancient  painting restoration with skip and borrow frame technique. This algorithm has 67 times faster than use Splitbregman algorithm on video.

\vspace{1 cm}

{\bf{Keywords:}} Numerical method, Total Variation, Image inpainting \\
\newpage