\thispagestyle{empty}
\vspace{2 cm}
{\huge \bf บทคัดย่อ}\addcontentsline{toc}{chapter}{บทคัดย่อ}

\vspace{2 cm}

\hspace{1cm} สำหรับงานวิจัยนี้ได้นำเสนอวิธีการเชิงตัวเลขที่ถูกพัฒนาขี้นสำหรับการซ่อมแซมภาพจิตรกรรมไทยและการลบบทบรรยายอนิเมะ โดยวิธีการเชิงตัวเลขนี้ใช้สำหรับการแก้ปัญหาด้วยตัวแบบการแปรผันรวมสำหรับการต่อเติมภาพที่ถูกพัฒนาขึ้้นโดยคณะวิจัยของคุณ Chan และคุณ Shen \cite{ref:rof-inpaint-chan-shen} โดยการซ่อมแซมภาพศิลปะไทยจะใช้วิธีการสปริทเบรกแมนที่ถูกพัฒนาขึ้นโดยขณะวิจัยของคุณ Goldstein และคุณ Osher \cite{ref:splitbergman-inpaint} พร้อมทั้งประยุกต์ใช้ร่วมกับวิธีพีระมิดรูปภาพที่ถูกคิดค้นด้วยคุณ Andelson และคณะ  \cite{ref:image-pyramid} และลดปริมาณการทำซ้ำในระดับความคมชัดเดิม พบว่าวิธีการที่พัฒนาขึ้นใช้เวลาประมวลผลน้อยกว่าวิธีการสปริทเบรกแมนเดิมประมาณ 7 เท่าและมีคุณภาพของภาพหลังการต่อเติมที่ดีขึ้นเมื่อวัดด้วยปริมาณ Peak Signal Noise Ration (PSNR) \cite{ref:PSNR} นอกจากนี้การลบบทบรรยายอนิเมะจะใช้วิธีการที่ใช้ในการซ่อมแซมภาพศิลปะไทย เมื่อนำมาพัฒนาต่อโดยใช้วิธีการข้ามเฟรมและยืมเฟรม พบว่าสามารถทำงานได้เร็วกว่าการใช้วิธีการสปริทเบรกแมนกับวิดีโอถึง 67 เท่า 

\vspace{1 cm}
{\bf{คำสำคัญ:}} การต่อเติมภาพ\\
\newpage