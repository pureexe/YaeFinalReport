\thispagestyle{empty}
\vspace{2 cm}
{\huge \bf บทคัดย่อ}\addcontentsline{toc}{chapter}{บทคัดย่อ}

\vspace{2 cm}

\hspace{1cm} ตัวแบบการแปรผันได้รับการยอมรับอย่างกว้างขวางเพื่อนำมาใช้ต่อเติมภาพ เนื่องจากสมบัติการอนุรักษ์เส้นของเร็กกิวลาร์ไรซ์เซชันแบบการแปรผันรวม อย่างไรก็ตาม การพัฒนาวิธีการเชิงตัวเลขที่มีประสิทธิภาพสำหรับสร้างคำตอบซึ่งสอดคล้องกับสมบัติดังกล่าวเป็นงานที่ท้าทาย เนื่องจากเร็กกิวลาร์ไรซ์เซชันแบบการแปรผันรวมมีสมบัติซึ่งหาอนุพันธ์ไม่ได้และไม่เป็นเชิงเส้น ในงานวิจัยนี้ เราพิจารณาปัญหาการแก้ตัวแบบดังกล่าว จากความสำเร็จของขั้นตอนวิธีการเชิงตัวเลขสำหรับปัญหาการกำจัดสัญญาณรบกวนออกจากภาพ เราได้นำเสนอ \break วิธีการสปริทเบรกแมน สำหรับแต่ละรอบของการทำซ้ำ การคำนวณของวิธีการนี้ต้องการแก้ 2 ปัญหาย่อย ในปัญหาย่อยแรก เราพบความยุ่งยากในการหาคำตอบแม่นตรง สำหรับปัญหาย่อยที่ 2 คำตอบได้นำเสนออยู่ในรูปแบบปิด ทั้งนี้ เราได้นำเสนอขั้นตอนวิธีเชิงตัวเลขแบบใหม่ (ขั้นตอนวิธีแรก) ที่ใช้การหาค่าต่ำสุดแบบสลับในกรอบความคมชัดหลายระดับเพื่อสร้างคำตอบเชิงตัวเลขที่รวดเร็วและแม่นยำ หลังจากนั้น เราได้ศึกษาปัญหาการลบบทบรรยายออกจากอนิเมะ เราพบว่าตัวแบบการต่อเติมภาพดังกล่าวสามารถนำมาปรับปรุงเพื่อให้ผลลัพธ์ที่น่าพอใจ ในการแก้ตัวแบบที่ปรับปรุง เรานำเสนอขั้นตอนวิธีการข้ามและการยืม (ขั้นตอนวิธีที่สอง) ซึ่งใช้ขั้นตอนวิธีแรกเพื่อกำจัดบทบรรยายอย่างมีประสิทธิภาพ การทดสอบบนภาพสังเคราะห์และภาพศิลปะไทยโบราณยืนยันว่าขั้นตอนวิธีแรกมีประสิทธิภาพสูงกว่าวิธีการพื้นฐาน นอกจากนี้เราพบว่า ขั้นตอนวิธีที่สองได้นำไปสู่ผลการลบบทบรรยายจากอนิเมะที่มีคุณภาพสูงอย่างรวดเร็ว

\vspace{1 cm}
{\bf{คำสำคัญ:}} ขั้นตอนวิธีเชิงตัวเลข, การแปรผันรวม, การต่อเติมภาพ, ภาพศิลปะไทย, อนิเมะ\\
\newpage