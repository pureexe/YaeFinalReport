\begin{figure}[H]
	\centering
	\begin{subfigure}{\ResultSubFigureWidth \linewidth}
		\centering
		\includegraphics[width=\ResultSubFigurePadding \linewidth]{image/result_ex1/fixpoint01.png}
	\end{subfigure}
	\begin{subfigure}{\ResultSubFigureWidth \linewidth}
		\centering
		\includegraphics[width=\ResultSubFigurePadding \linewidth]{image/result_ex1/fixpoint02.png}
	\end{subfigure}
	\begin{subfigure}{\ResultSubFigureWidth \linewidth}
		\centering
		\includegraphics[width=\ResultSubFigurePadding \linewidth]{image/result_ex1/fixpoint03.png}			
	\end{subfigure}
	\begin{subfigure}{\ResultSubFigureWidth \linewidth}
		\centering
		\includegraphics[width=\ResultSubFigurePadding \linewidth]{image/result_ex1/fixpoint04.png}			
	\end{subfigure}
	\begin{subfigure}{\ResultSubFigureWidth \linewidth}
		\centering
		\includegraphics[width=\ResultSubFigurePadding \linewidth]{image/result_ex1/fixpoint05.png}			
	\end{subfigure}
	\caption{ผลการซ่อมแซมจากวิธีการทำซ้ำแบบจุดตรึง}
\end{figure}