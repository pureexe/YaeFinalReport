\section{การประยุกต์ของการต่อเติมภาพ}
\subsection{การซ่อมแซมภาพจิตรกรรมไทยโบราณ}

\hspace{1cm} ภาพจิตรกรรมไทย คือ ภาพเขียนที่มีเอกลักษณ์ความเป็นศิลปะไทยซึ่งโดดเด่นและแตกต่างจากภาพเขียนของชนชาติอื่น   ช่างไทยในอดีตได้สร้างสรรค์ลวดลายและสีสันบนภาพวาดเพื่อสะท้อนประเพณีและวัฒนธรรมในสังคมไทยที่เกี่ยวกับศาสนา ประวัติศาสตร์ โบราณคดี ชีวิตความเป็นอยู่ วัฒนธรรมการแต่งกาย ตลอดจนการแสดงการเล่นพื้นเมืองต่าง ๆ ของแต่ละยุคสมัย 
	
\hspace{1cm} อย่างไรก็ตาม ภาพจิตรกรรมไทยโบราณจำนวนไม่น้อยได้เสื่อมสลายตามกาลเวลา และรอคอยการซ่อมแซมจากช่างในสมัยปัจจุบันที่ต้องไม่สร้างความเสียหายให้กับภาพเขียนเพิ่มขึ้นมากกว่าเดิม  ที่ผ่านมาภาพที่ซึ่งรับซ่อมแซมมาแล้วจำนวนไม่น้อยได้รับความเสียหายหลังจากการซ่อมแซม ถึงแม้สภาพโดยรวมของภาพจิตรกรรมเดิมยังคงอยู่ แต่รายละเอียดในตัวภาพเขียนได้เปลี่ยนไป ก่อให้เกิดความเสียหายที่ประเมินค่าไม่ได้ 
	
\hspace{1cm} การซ่อมแซมภาพจิตรกรรมไทยโบราณโดยใช้ขั้นตอนวิธีการต่อเติมภาพเป็นขั้นตอนของการซ่อมแซมแบบหนึ่งซึ่งไม่ก่อให้เกิดความเสียหายใด ๆ กับภาพเดิม เนื่องจากเป็นการซ่อมแซมโดยการใช้ขั้นตอนวิธีเชิงตัวเลขบนภาพ ดิจิตัลซึ่งเป็นสำเนาของภาพเดิม ด้วยเหตุผลดังกล่าว ผู้วิจัยได้เล็งเห็นว่าการซ่อมแซมภาพจิตรกรรมไทยโบราณมีความจำเป็นเร่งด่วน เนื่องจากภาพที่ได้รับการซ่อมแซมด้วยการต่อเติมภาพสามารถนำไปใช้ประกอบการตัดสินใจเพื่อวางแผนก่อนการลงมือซ่อมแซมภาพเขียนจริงได้ นอกจากนี้ ขั้นตอนวิธีการต่อเติมภาพสามารถนำไปใช้สร้างแอปพลิเคชันบนโทรศัพท์เคลื่อนที่เพื่อในไปใช้เป็นข้อมูลในการเข้าชมภาพเขียนเดิมที่ยังไม่ได้รับการซ่อมแซมและภาพเขียนที่ได้รับการซ่อมแซมโดยวิธีการทางคณิตศาสตร์จากแอปพลิเคชันที่พัฒนาขึ้น
	
\hspace{1cm} รูปที่ \ref{figure:inpaint-thaiairt-explain} แสดงตัวอย่างภาพจิตรกรรมไทย\footnote{ภาพถ่ายที่วัดภูมินทร์ อำเภอเมือง จังหวัดน่าน; ภาพจาก http://topicstock.pantip.com/camera/topicstock/2009/02/O7514399/O7514399.html สืบค้นเมื่อวันที่ 23 กันยายน 2561} ที่ต้องได้รับการซ่อมแซมบนบริเวณแขนเสื้อของรูปวาดผู้ชายที่มีส่วนของสีแดงเดิมหลุดหายไป ทั้งนี้ในการซ่อมแซมภาพโดยการต่อเติมภาพ เราจะเริ่มด้วยการสร้างโดเมนต่อเติมบนบริเวณสีพื้นผิวปูนที่แขนเสื้อ จากนั้นจึงนำขั้นตอนวิธีการต่อเติมภาพเพื่อซ่อมแซมภาพบริเวณนั้นให้เป็นสีแดง 

\begin{figure}[h]
	\[
	\includegraphics[width=0.6\linewidth]{image/explain_art/fig2a.jpg}
	\]
	\caption{ภาพจิตรกรรมไทยที่วัดภูมินทร์ อำเภอเมือง จังหวัดน่าน}
	\label{figure:inpaint-thaiairt-explain}
\end{figure}

\subsection{การลบบทบรรยายจากอนิเมะ}
\hspace{1cm}อนิเมะคือวิดีโอภาพวาดการ์ตูนสไตล์ญี่ปุ่นซึ่งเป็นที่นิยมของเยาวชนไทย ในการรับชมอนิเมะ แม้ว่าเยาวชนไทยสามารถรับชมอนิเมะด้วยบทพากย์เสียงภาษาไทยอย่างถูกต้องตามลิขสิทธิ์การเผยแพร่ แต่ก็สูญเสียอรรถรสของการรับชมจากบทบรรยายแบบแข็ง\footnote{บทยรรยายที่ไม่สามารถปิดหรือเปิดได้} (hardsub) ที่เป็นภาษาต่างประเทศในบริเวณด้านล่างของจอภาพ อย่างไรก็ตาม ในการซ่อมแซมอนิเมะด้วยการลบบทบรรยายภาษาต่างประเทศจึงเป็นงานที่ยุ่งยากและท้าท้ายมาก เนื่องจาก
\begin{itemize}
	\item [(1)] อนิเมะเป็นวิดีโอซึ่งแสดงผลประมาณ 24 เฟรม(ภาพ)ต่อวินาที
	\item [(2)] แต่ละเฟรมอาจมีหรืออาจไม่มีบทบรรยายก็ได้
	\item [(3)] แต่ละเฟรมอาจมีหรืออาจไม่มีบทบรรยายเดียวกันก็ได้
	\item [(4)] แต่ละเฟรมเป็นการแสดงผลภาพสีที่มีระดับความคมชัดสูง ขนาดมากถึง $1920\times1080$ พิกเซล
\end{itemize}

\hspace{1cm} ด้วยความท้าทายข้างต้น การพัฒนาขั้นตอนวิธีการต่อเติมภาพที่สามารถกำหนดโดเมนต่อเติมเชิงอัตโนมัติให้กับแต่ละเฟรมและประมวลผลได้แม่นยำจนการลบบทบรรยายสามารถทำงานได้แบบเรียลไทม์จึงเป็นสิ่งจำเป็นที่หลีกเลี่ยงไม่ได้
	
\hspace{1cm} รูปที่ \ref{figure:inpaint-anime-explain} แสดงตัวอย่าง 1 เฟรมของอนิเมะ\footnote{ภาพจาก https://www.samehadaku.tv/2018/07/grand-blue-episode-1-subtitle-indonesia.html สืบค้นเมื่อวันที่ 23 กันยายน 2561}ที่มีบทบรรยายแบบแข็ง ที่ต้องซ่อมแซมด้วยการลบบทบรรยายออก  ทั้งนี้ในการลบบทบรรยายออกจากเฟรมโดยใช้การต่อเติมภาพ เราจะเริ่มด้วยการสร้างโดเมนต่อเติมแบบอัตโนมัติในบริเวณที่มีบทบรรยาย จากนั้นจึงนำขั้นตอนวิธีการต่อเติมภาพแบบเร็วเพื่อลบบทบรรยายออกจากเฟรม 

% ถ้าหน้าเลื่อนกลับมาปรับขนาดภาพเป็น 0.6
\begin{figure}[h]
    \[
    \includegraphics[width=0.3\linewidth]{image/explain_art/fig3.png}
    \]
    \caption{1 เฟรมของอนิเมะที่มีบทบรรยายแบบแข็ง}
    \label{figure:inpaint-anime-explain}
\end{figure}
	
%\hspace{1cm} โครงการวิจัยนี้ ผู้วิจัยมีเป้าหมายสำคัญคือการพัฒนาขั้นตอนวิธีการต่อเติมภาพแบบเร็วและแม่นยำชนิดใหม่เพื่อนำไปใช้สำหรับซ่อมแซมภาพจิตรกรรมไทยและการลบบทบรรยายออกจากอนิเมะ