\section{ตัวแบบเชิงแปรผันสำหรับต่อเติมภาพเฉดเทา}\label{inpaint-model-grayscale}

\hspace{1cm} ในการต่อเติมภาพเฉดสีเทา Chan และ Shen \cite{ref:rof-inpaint-chan-shen} ได้นำเสนอตัวแบบเชิงการแปรผัน (variational model) ที่ใช้เร็กกิวลาร์ไรซ์เซชันแบบการแปรผันรวม (Total variation based regularization) โดยพัฒนาต่อจากตัวแบบ ROF สำหรับการกำจัดสัญญาณรบกวน \cite{ref:ROF-template} ซึ่งตัวแบบเชิงการแปรผันนี้กำหนดโดย
\begin{align}
    \min_{u} \{ \mathcal{J}(u) = \frac{1}{2} \int_{\Omega}\lambda (u-z)^2 d\Omega +  \int_{\Omega}  |\nabla u|  d\Omega \}
\label{e1}
\end{align}

เมื่อ 
\begin{align}
    \lambda=\lambda(\mathbf{x}) = \left \{ \begin{array}{ll}  \lambda_0, & x \in \Omega \setminus D \\ 0, & x \in D  \end{array} \right . 
    \label{e2}
\end{align}
แทนพารามิเตอร์เร็กกิวลาร์ไรซ์เซชัน (regularization parameter) และ $\lambda_0 >0$

\hspace{1cm} โดยแคลคูลัสของการแปรผัน (Calculus of variations) จะได้สมการออยเลอร์ลากรางจ์ที่เกี่ยวข้องกับ (\ref{e1}) เป็น 
\begin{align}
    \left \{ \begin{array}{ll}  - \nabla \cdot  \Big( \dfrac{\nabla u}{|\nabla u|} \Big) + \lambda (u-z) = 0,  & \hspace{1cm} \mathbf{x} \in (1,n)^2 \\ \dfrac{\partial u}{\partial \boldsymbol{n}} = 0, & \hspace{1cm} x \in \partial \Omega \end{array} \right . 
    \label{e3}
\end{align}
เมื่อ $\boldsymbol{n}$ แทนเวกเตอร์หน่วยที่ตั้งฉากกับของของภาพ
