\section{การวัดคุณภาพของภาพที่ผ่านกระบวนการต่อเติม}

\hspace{1cm} การประเมินคุณภาพของการต่อเติมภาพของวิธีการเชิงตัวเลข โดยการวัดค่าคุณภาพของภาพ ในโครงงานวิจัยนี้จะใช้ 2 ค่าคือ จะใช้ค่า Peak Signal Noise Ratio (PSNR) \cite{ref:PSNR} และ Structral Similarity (SSIM) \cite{ref:SSIM} 

\subsection{Peak Signal Noise Ratio}

\hspace{1cm} ค่า PSNR ใช้สำหรับวัดค่าคุณภาพของแต่ละพิกเซลในภาพ มีหน่วยเป็น เดซิเบล (dB) ซึ่งมีช่วงค่าอยู่ในระหว่าง 0 ถึง อนันต์ โดยค่า PSNR ที่ได้ยิ่งเข้าใกล้สู่อนันต์แปลว่าคุณภาพที่ได้ในแต่ละพิกเซลนั้นยิ่งมีคุณภาพที่ดี ซึ่งสามารถคำนวณค่า PSNR ได้ดังนี้
\begin{align}
    \text{PSNR}  = 10 \cdot log_{10} ( \frac{1}{\sqrt{\text{MSE}}} )
\end{align}
\begin{itemize}
	\item[$\bullet$] MSE คือค่าคลาดเคลื่อนกำลังสองเฉลี่ยของภาพ โดยที่ MSE = $\bigg( \frac{1}{nx \times ny} \sum (u - \bar{u})^2  \bigg)$
	\item[$\bullet$] $u$ แทนภาพต้นฉบับ
	\item[$\bullet$] $\tilde{u}$  แทนภาพต้นฉบับ และภาพที่ได้จากการซ่อมแซมโดยวิธีเชิงตัวเลข
\end{itemize}

\subsection{Structral Similarity}
\hspace{1cm} ค่า SSIM ใชสำหรับวัดคุณภาพของเค้าโครงในภาพผลลัพธ์ว่ามีเค้าโครงคล้ายกับภาพของเดิมมากเพียงใด โดยค่า SSIM จะมีค่าอยู่ระหว่าง 0 ถึง 1 หากค่าที่ได้ยิ่งเข้าใกล้สู่ 1 นั้นคือเค้าโครงใกล้เคียงกับภาพเดิมก่อนทำการต่อเติมภาพ ซึ่งสามารถคำนวณ SSIM ได้ดังนี้

\begin{align*}
	\text{SSIM}(u,\tilde{u}) = \frac{(2\mu_u\mu_{\tilde{u}} + 0.0001)(2\sigma_{u\tilde{u}} + 0.0009)}{(\mu_u^2+\mu_{\tilde{u}}^2+0.0001)(\sigma_u^2+\sigma_{\tilde{u}}^2+0.0009)}
\end{align*}

\begin{itemize}
	\item[$\bullet$] $u$ แทนภาพต้นฉบับ
	\item[$\bullet$] $\tilde{u}$  แทนภาพต้นฉบับ และภาพที่ได้จากการซ่อมแซมโดยวิธีเชิงตัวเลข
	\item[$\bullet$] $\mu_u$ คือค่าเฉลี่ยของ $u$
	\item[$\bullet$] $\mu_{\tilde{u}}$ คือค่าเฉลี่ยของ $\tilde{u}$
	\item[$\bullet$]  $\sigma_u$ คือความแปรปรวนของ $u$ 
	\item[$\bullet$] $\sigma_{\tilde{u}}$ คือความแปรปรวนของ $\tilde{u}$
\end{itemize}
