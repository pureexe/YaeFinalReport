\section{วิธีการทำซ้ำสำหรับระบบสมการเชิงเส้น}

\hspace{1cm} โครงงานวิจัยเรื่องนี้จะมีการทำซ้ำด้วยกัน 2 วิธีนั่นคือ วิธีจุดตรึง โดยวิธีจุดตรึงนี้จะใช้ในส่วนวิธีเชิงตัวเลขสำหรับการต่อเติมภาพด้วยวิธีการเดินเวลาแบบชัดแจ้งและการทำซ้ำแบบจุดตรึง และอีกวิธีการคือวิธีการเกาส์-ไซเดล ซึ่งจะใช้ในวิธีเชิงตัวเลขสปริทเบรกแมน 

\subsection{วิธีการทำซ้ำจุดตรึง}
\hspace{1cm} วิธีการทำซ้ำจุดตรึง (fixed-point iteration) นิยมใช้แก้ปัญหาที่อยู่ในรูป $f(x) = 0$ โดยจะทำการจัดรูปด้วยวิธีการทางพีชคณิตให้ได้สมการอยู่ในรูป $x = g(x)$ จากนั้นจึงทำซ้ำ

\begin{align*}
    x_{i+1} = g(x_i), \hspace{1.5cm} i=0,1,2,...
\end{align*}

เมื่อ $i$ มากเกินกว่าจำนวนรอบการทำซ้ำที่กำหนด หรือ $|x_{i+1} - x_{i}|$ น้อยกว่าค่าที่กำหนดจะหยุดการทำซ่้ำ แล้วจะได้ว่า $x_i$ เป็นคำตอบของปัญหา


\subsection{วิธีการเกาส์-ไซเดล}
\hspace{1cm} สำหรับวิธีการเกาส์-ไซเดล นิยมใช้ในการแก้ระบบสมการที่อยู่ในรูป $Ax=b$ เมื่อ $A$ คือเมทริกซ์ขนาด $m$ x $n$, $b$ และ $x$ เป็นเวคเตอร์ขนาด $m$ x $1$