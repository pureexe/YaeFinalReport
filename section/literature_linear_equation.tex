\section{วิธีการทำซ้ำสำหรับระบบสมการเชิงเส้น}


\hspace{1cm}  ในหัวข้อนี้จะแนะนำถึงวิธีการทำซ้ำเพื่อแก้ปัญหาระบบสมการเชิงเส้น สำหรับโครงงานวิจัยเรื่องนี้มีการแก้ปัญหาของระบบสมการเชิงเส้น เพื่อทำการแก้สมการเชิงอนุพันธ์ที่ถูกเปลี่ยนให้เป็นระบบสมการเชิงเส้นแล้ว โดยระบบสมการเชิงเส้นที่จะทำการแก้นั้นอยู่ในรูปของ
\begin{align}
    Ax = b
    \label{equation:linearquation_system}
\end{align}
เมื่อ $x \in \mathbb{R}^{N}$ และ $A$ เป็นเมทริกซ์ขนาด $N \times N$ โดยการทำซ้ำนี้จะเริ่มจากค่าประมาณเริ่มต้น (intial approximation) $x^{(0)}$ และทำการสร้างลำดับ $\{ x^{(k)} \}_{k=1}^\infty$ จากความสัมพันธ์
\begin{align}
    x^{(k)} = Tx^{(k-1)} + c
\end{align}
เมทริกซ์ความสัมพันธ์ $T$ และเวคเตอร์ $c$ มาจากการแบ่ง $A = M-N$ ของเมทริกซ์ $A$ เมื่อ $M$ เป็นเมทริกซ์ไม่เอกฐาน โดยแยกระบบเดิม \ref{equation:linearquation_system} ออกเป็น
\begin{align}
    Ax = (M - N)x = b
\end{align}
นั่นคือ
\begin{align}
    x = (M^{-1} N)x + M^{-1}b = Tx + c
\end{align}
เมื่อ $T = M^{-1}N$ และ $c = M^{-1}b$
\hspace{1cm} โครงงงานวิจัยนี้ทางผู้วิจัยได้เลือกใช้วิธีเกาส์-ไซเดลซึ่งเป็นวิธีการที่พัฒนาต่อมาจากวิธีการจาโคบี จึงขอนำเสนอทั้งสองวิธีการ ดังนี้

\subsection{วิธีการจาโคบี}
วิธีการจาโคบีจะแก้สมการที่ $i$ ของ $Ax = b$ โดยหา $x_i$ ซึ่งกำหนดโดย
\begin{align}
    x_i = \sum_{\substack{j=1 \\ j\neq i}}^{N} \Big( \frac{- a_{ij}x_j}{a_{ii}} \Big) + \frac{b_i}{a_{ii}} \hspace{1cm}\text{ เมื่อ } i=1,...,N
\end{align}
ให้ $x^{(k-1)}$ สำหรับทุก $k \geq 1$ ซึ่งก่อกำเนิดโดย
\begin{align}
    x_i^{(k)} = \sum_{j=1}^{N}  \Big( \frac{- a_{ij}x_j^{(k-1)}}{a_{ii}} \Big) + \frac{b_i}{a_{ii}} \hspace{1cm}\text{ เมื่อ } i=1,...,N
\end{align}
ซึ่งจำเป็นที่ $a_{ii} \neq 0$ สำหรับ $i = 1,..., N $ แต่ถ้ามีอย่างน้อยหนึ่ง $a_{ii} = 0$ และระบบไม่เอกฐาน ก็สามารถสับเปลี่ยนลำดับเพื่อให้ไม่มี $a_{ii}$ ที่เป็น 0 ได้ และการเขียน $Ax = b$ เป็น $x = Tx +c$ จะทำการเปลี่ยน $A$ เป็น $A = D - L - U$ เมื่อ $D$ เป็นเมทริกซ์แทยงมุมของ $A$, $-L$ เป็นสามเหลี่ยมส่วนล่างของ $A$ และ $-U$ เป็นสามเหลี่ยมส่วนบนของ $A$ จึงได้ว่า
\begin{align}
    Ax = (D - L - U)x = b    
\end{align}
หรือ
\begin{align}
    x = D^{-1} (L+U)x + D^{-1}b
\end{align}
เมื่อทำการแบ่งเมทริกซ์เป็น $A = M - N$ โดยที่ $M = D$ และ $N = L + U$ แล้วจะได้ว่าเมทริกซ์สำหรับวิธีการจาโคบีคือ
\begin{align}
    x^(k) = T_{j}x^(x-1) + c_J
\end{align}
เมื่อ $T_J = D^{-1}(L+U)$ และ $c_J = D^{-1}b$

\textbf{ขั้นตอนวิธีจาโคบี}

ขั้นตอนวิธีของจาโคบีเพื่อหาค่าใกล้เคียงของคำตอบ $Ax = b$ จะให้ค่าใกล้เคียงของคำจอบเริ่มต้นเป็น $x^{(0)}$ ให้จำนวนรอบการทำซ้ำสูงสุดเป็น $IMAX$ และให้ค่าความคลาดเคบื่อนเป็น $\epsilon > 0$

\begin{algorithm}[H]
    \label{algorithm:subtitle_skipnborrowframe}
    \caption{ขั้นตอนวิธีจาโคบี}  
    \SetAlgoNoLine
    \SetKwFunction{FMain}{$[x] \longleftarrow Jacobi$}
    \SetKwProg{Fn}{}{}{}
    \Fn{\FMain{$A , b, x^{(0)}, IMAX, \epsilon$}}{
        1. ให้ $k=1, N = size(x^{(0)}), \text{done = False}$\\
        2. ถ้า done = False ทำซ้ำขั้นตอนที่ 3 และ 4\\
        3. $x_i^{(x)} = \sum_{j=1}^{(k-1)} \Big( \frac{- a_{ij} x_j^{(k-1)} }{ a_ii } + \frac{b_i}{a_{ii}} \Big) $ \\
        4. \uIf{ $ || b - Ax^{(k)} || < \epsilon $ หรือ $ || x^{(k)} - x^{(k-1)} || < \epsilon $ หรือ $k \geq \epsilon$}{
            ให้ done = True และ $x = x^{(k)}$
        }\Else{
            ให้ $k = k + 1$
        }
    }	
\end{algorithm}

\subsection{วิธีการเกาส์-ไซเดล}
จากวิธีการจาโคบีมีคำนวณคำ $x_1^{(k)},x_2^{(k)}, ... , x_{i-1}^{(k)}$ ซึ่งสามารถเพิ่มประสิทธิภาพได้ด้วยการเปลี่ยนสมการของ $x_i^{(k)}$ เป็น 
\begin{align}
    x_i^{(k)} = \frac{-\sum_{j=1}^{i-1} a_{ij}x_{j}^{(k)} - \sum_{j=i+1}^{N} a_{ij}x_{j}^{(k-1)} + b_i }{a_{ii}}
    \label{equation:gauss_seidel_main}
\end{align}
วิธ๊การนี้เรียกว่า วิธีการเกาส์-ไซเดล ซึ่งสามารถเขียนสมการใหม่ได้เป็น
\begin{align}
    a_{ii} x_{i}^{(k)} + \sum_{j=1}^{i-1} a_{ij} x_{j}^{(k)} = - \sum_{j=i+1}^{N} a_{ij} x_j^{(k-1)} + b_i
\end{align}
จะได้รูปแบบเมทริกซ์ของวิธีเกาส์-ไซเดลเป็น
\begin{align}
    (D - L) x^{(k)} = U x^{(k-1)} + b
\end{align}
ซึ่งสมมูลกับ
\begin{align}
    x^{(k)} = T_{GS} x^{(k-1)} + c_{GS}
\end{align}
เมื่อ $T_{GS} = (D - L)^{-1} U$ และ $c_{GS} = (D - L)^{-1} b$ นั่นคือเกาส์-ไซเดล มีพื้นฐานมาจากการแยกเมทริกซ์ด้วย $M = D - L$ และ $N = U$

\textbf{ขั้นตอนวิธีเกาส์-ไซเดล}

ขั้นตอนวิธีเกาส์-ไซเดล เหมือนกับขั้นตอนวิธีจาโคบี แต่เปลี่ยนขั้นที่ 3 เป็นสมการที่     \ref{equation:gauss_seidel_main} 