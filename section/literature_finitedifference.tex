\section{วิธีการไฟไนต์ดิฟเฟอเรนจ์เบื้องต้น}
\hspace{1cm} วิธีการไฟไนต์ดิฟเฟอเรนจ์ (finite difference method) เป็นวิธีการหนึ่งที่นิยมใช้ในการประมาณคำตอบของปัญหาค่าขอบ โดยวิธีนี้จะทำการเปลี่ยนอนุพันธ์ในสมการเป็นการประมาณไฟไนต์ดิฟเฟอเรนจ์ (finite difference approximation) และทำการแก้ปัญหาค่าขอบด้วยขั้นตอน 3 ขั้นตอนดังนี้

\begin{enumerate}
    \item ทำให้คำตอบในโดเมน $\Omega$ อยู่ในรูปแบบไม่ต่อเนื่อง โดยแต่ละจุดที่ไม่ต่อเนื่องนี้จะมีชื่อเรียกว่าโหนด (node)
    \item ประมาณอนุพันธ์ทั้งหมดด้วยไฟไนต์ดิฟเฟอเรนจ์ ซึ่งขั้นตอนนี้จะทำให้สมการเชิงอนุพันธ์ถูกเปลี่ยนเป็นสมการพีชคณิตทำให้สามารถแก้สมการได้ง่ายขึ้น
    \item ทำการแก้สมการพีชคณิตด้วยวิธีการต่างๆ
\end{enumerate}