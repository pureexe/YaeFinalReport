\section{วิธีการไฟไนต์ดิฟเฟอเรนจ์เบื้องต้น}
\hspace{1cm} วิธีการไฟไนต์ดิฟเฟอเรนจ์ (finite difference method)  เป็นวิธีการเชิงตัวเลขที่พัฒนาขึ้นเพื่อแก้ไขปัญหาค่าขอบ ขั้นตอนของวิธีการไฟไนต์ดิฟเฟอเรนซ์สำหรับการแก้ปัญหาค่าขอบประกอบด้วยสามขั้นตอนสำคัญดังนี้

\begin{enumerate}
    \item ดิสครีตไทซ์ (discretize) โดเมนของผลเฉลย (solution domain) ออกเป็นช่องตาราง (mesh) ของจุดกริด (grid point) ที่ต้องการหาผลเฉลยเชิงตัวเลข
    \item ประมาณอนุพันธ์ที่ปรากฏในปัญหาค่าขอบด้วยการประมาณแบบไฟไนต์ดิฟเฟอเรนซ์ (finite difference approximation) ในขั้นตอนนี้ การประมาณดังกล่าวจะนำไปสู่ระบบสมการเชิงเส้น หรือระบบสมการไม่เป็นเชิงเส้นขนาดใหญ่ที่สมนัยกับปัญหาค่าขอบตั้งต้น
    \item แก้ระบบสมการเชิงเส้นหรือระบบสมการไม่เป็นเชิงเส้นขนาดใหญ่ที่เกิดขึ้นในขั้นตอนที่ 2 เพื่อกำหนดผลเฉลยเชิงตัวเลข
\end{enumerate}

\hspace{1cm} กำหนดให้ $u(x)$ แทนฟังก์ชันค่าจริงและเป็นฟังก์ชันราบเรียบ (smooth function) นั่นคือ $u$ สามารถหาอนุพันธ์ได้หลายครั้ง โดยแต่ละครั้ง อนุพันธ์ที่หาได้เป็นฟังก์ชันที่ถูกนิยามอย่างดี (well-defined) และมีขอบเขตเหนือช่วงที่มีจุดที่สนใจ $\bar{x}$

\hspace{1cm} ในการประมาณ $u'(\bar{x})$ โดยใช้ค่าของ $u$ ที่เกิดจากจุดที่อยู่ในบริเวณใกล้เคียงกับ $\bar{x}$ สามารถใช้สูตรการประมาณแบบไฟไนต์ดิฟเฟอเรนซ์ที่ถูกกำหนดได้ดังต่อไปนี้
\begin{enumerate}
    \item สูตรฟอร์เวิร์ดดิฟเฟอเรนซ์ (forward-difference formular) 
    \begin{align*}
        D_{+}u(\bar{x}) = \frac{u(\bar{x}+h) - u(\bar{x})}{h}    
    \end{align*}
    \item สูตรแบ็คเวิร์ดดิฟเฟอเรนซ์ (backward-difference formular) 
    \begin{align*}
        D_{-}u(\bar{x}) = \frac{u(\bar{x}) - u(\bar{x}-h)}{h}    
    \end{align*}
    \item สูตรเซ็นทรัลดิฟเฟอเรนซ์ (forward-difference formular) 
    \begin{align*}
        D_{0}u(\bar{x}) = \frac{u(\bar{x}+h) - u(\bar{x}-h)}{2h}    
    \end{align*}
\end{enumerate}
โดยที่ $h$ เป็นจำนวนจริงที่มีค่าน้อยๆ ซึ่ง $h>0$ 

\begin{Example}
    พิจาราณาปัญหาค่าขอบดังต่อไปนี้
    \begin{align}
        \left \{ \begin{array}{ll}   -\Delta u = f & x \in \Omega =	(0,1)^2	 \\
        u = 0 & \hspace{1cm} x \in \partial \Omega \end{array} \right . 
        \label{equation:demo_discrete}
    \end{align}
    \hspace{1cm} กำหนดให้ $m_x$, $m_y$ เป็นจำนวนนับแทนจำนวนจุดกริดในทาง $x$ และทาง $y$ ตามลำดับ และ
    \begin{align*}
        \Omega_{h}=\left\{ \mathbf{x} \in \Omega | \mathbf{x}=(x_i,y_j)^{\top} = (ih_x,jh_y), 1 \leq i \leq m_x, 1 \leq j \leq m_y  \right\}
    \end{align*}
    แทนโดเมนของผลเฉลยที่ถูกดิสครีตไทซ์ซึ่งประกอบด้วย $(m_x+1) \times (m_y+1)$ เซลล์ (cell) แต่ละเซลล์มีขนาด $h_x \times h_y$ โดยที่
    \begin{align*}
        h = (h_x,h_y)^{\top} = (\frac{1}{m_x+1},\frac{1}{m_y+1})^{\top}
    \end{align*}

    \hspace{1cm} กำหนดให้ $u(x,y)$ แทนคำตอบของปัญหา (\ref{equation:demo_discrete}), $(f)_{i,j}$ แทน $f(x_i,y_j)$ และ $(u)_{i,j}$ แทนฟังก์ชันกริด (grid function) ซึ่งสัมพันธ์กับคำตอบแม่นตรงที่จุดกริด $(x_i,y_j)^{\top}$ 
    
    \hspace{1cm} การประมาณแบบไฟไนต์ดิฟเฟอเรนซ์สำหรับตัวดำเนินการลาปลาซ $\Delta u$ สามารถถูกนำเสนอด้วยสเตนซิล 5 จุด (5-point stencil) ได้ดังนี้
    \begin{align}
        \nonumber \Delta u(x,y) = &\frac{1}{h_x^2} \Big( u(x+h_x,y) + u(x-h_x,y) - 2 u(x,y) \Big) \\&+ \frac{1}{h_y^2} \Big( u(x,y+h_y) + u(x,y-h_y) - 2 u(x,y) \Big) 
        \label{equation:5point-stencil}
    \end{align}

    เราจะทำการแทนที่การดำเนินการลาปลาซใน  (\ref{equation:demo_discrete}) ด้วยสูตรไฟไนต์ดิฟเฟอร์เรนซ์ได้ดังนี้
    \begin{align}
        - \Big( \frac{1}{h_x^2} \big( (u)_{i-1,j} - 2(u)_{i,j} + (u)_{i+1,j}\big) + \frac{1}{h_y^2} \big( (u)_{i,j-1} - 2(u)_{i,j} + (u)_{i,j+1}\big) \Big) = (f)_{i,j}
        \label{equation:extracted_finite}
    \end{align}
    ภายใต้เงื่อนไขขอบ $(u)_{i,j} = 0$ เมื่อ $i = 0$ หรือ $m_x$, $j = 0$ หรือ $m_y$ ซึ่ง (\ref{equation:extracted_finite}) เป็นระบบสมการเชิงเส้นซึ่งสามารถแก้ได้ด้วยวิธีการเกาส์ไซเดลที่จะพูดถึงในหัวข้อถัดไป
\end{Example}