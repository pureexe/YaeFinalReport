\section{วิธีการไฟไนต์ดิฟเฟอเรนจ์เบื้องต้น}
ไฟไนต์ดิฟเฟอร์เรนจ์ (Finite Difference) คือวิธีการสำหรับการประมาณค่าอนุพันธ์เมื่อใช้วิธีเชิงตัวเลข

\subsection{การหาอนุพันธ์}
สำหรับการหาคาอนุพันธ์ในโครงงานวิจัยนี้จะใช้วิธีการฟอร์เวิร์ดดิฟเฟอร์เรนจ์(Forward Difference) และใช้ค่าขอบแบบ neuman

\subsection{การหาแกรเดียน}
สำหรับการหาแกรเดียน (Gradient) จะใช้วิธีฟอร์เวิร์ดดิฟเฟอร์เรนจ์ในแนวแกน x และแนวแกน y คำตอบที่ได้จะเป็นเวคเตอร์ของอนุพันธ์แนวแกน x และอนุพันธ์แนวแกน y


\subsection{การหาไดเวอร์เจน}
สำหรับไดเวอร์เจน (Divergence) จะเป็นการหาผลรวมของอนุพันธ์ในแต่ละแกนของเวคเตอร์ นั่นคือ 

\subsection{การหาลาปาเชียน}
สำหรับลาปาเซียน (Lapacian) นั่นคือการทำหาไดเวอร์เจรบนเวคเตอร์ที่หาแกรเดียนแล้ว ทั้งนี้มีค่าเท่ากับ

\begin{align*}
    \triangle u_{i,j} = u_{i-1,j} + u_{i+1,j} + u_{i,j-1} + u_{i,j+1} - 4u_{i,j} 
\end{align*}