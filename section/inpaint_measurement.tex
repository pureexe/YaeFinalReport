\section{การวัดประสิทธิภาพของภาพที่ผ่านกระบวนการต่อเติม}	

\hspace{1cm} การประเมินคุณภาพของการต่อเติมภาพของวิธีการเชิงตัวเลข จะใช้ค่า PSNR \cite{ref:PSNR} และ SSIM \cite{ref:SSIM} โดย
	
\begin{align*}
	\text{PSNR}  &= 10 \cdot log_{10} ( \frac{1}{\sqrt{\text{MSE}}} ) \\
	\text{SSIM}(u,\tilde{u}) &= \frac{(2\mu_u\mu_{\tilde{u}} + 0.0001)(2\sigma_{u\tilde{u}} + 0.0009)}{(\mu_u^2+\mu_{\tilde{u}}^2+0.0001)(\sigma_u^2+\sigma_{\tilde{u}}^2+0.0009)}
\end{align*}
	
\begin{itemize}
	\item[$\bullet$] MSE คือค่าคลาดเคลื่อนกำลังสองเฉลี่ยของภาพ โดยที่ MSE = $\bigg( \frac{1}{nx \times ny} \sum (u - \bar{u})^2  \bigg)$
	\item[$\bullet$] $u$ แทนภาพต้นฉบับ
	\item[$\bullet$] $\tilde{u}$  แทนภาพต้นฉบับ และภาพที่ได้จากการซ่อมแซมโดยวิธีเชิงตัวเลข
	\item[$\bullet$] $\mu_u$ คือค่าเฉลี่ยของ $u$
	\item[$\bullet$] $\mu_{\tilde{u}}$ คือค่าเฉลี่ยของ $\tilde{u}$
	\item[$\bullet$]  $\sigma_u$ คือความแปรปรวนของ $u$ 
	\item[$\bullet$] $\sigma_{\tilde{u}}$ คือความแปรปรวนของ $\tilde{u}$
\end{itemize}
	
\textbf{หมายเหตุ:}
\begin{itemize}
	\item [(1)] ถ้า $\tilde{u} \longrightarrow u $ แล้ว PSNR $\longrightarrow \infty$ หมายถึง ภาพที่ซ่อมแซม $\tilde{u}$ มีคุณภาพดี
	\item [(2)] ถ้า $SSIM(u,\tilde{u}) = 1 $
	หมายถึง ภาพที่ซ่อมแซม $\tilde{u}$ มีคุณภาพดี
\end{itemize}
	
