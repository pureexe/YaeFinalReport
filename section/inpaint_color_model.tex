\section{ตัวแบบการต่อเติมภาพสีที่ใช้การแปรผันรวม}\label{inpaint-model-color}

\hspace{1cm} ต่อไปเราจะพิจารณาภาพสีในระบบสี RGB นั่นคือ เราสมมติว่า	
\begin{align*}
	\boldsymbol{u} = (u_1,u_2,u_3)^{\top},\ \boldsymbol{z} = (z_1,z_2,z_3)^{\top} : \Omega  \rightarrow V^3
\end{align*}
	
\noindent เมื่อ $u_1,u_2,u_3: \Omega  \rightarrow V$ และ $z_1,z_2,z_3: \Omega  \rightarrow V$ แทนภาพในเฉดสีแดง สีเขียว และสีน้ำเงินของ $\boldsymbol{u},\boldsymbol{z}$ ตามลำดับ 
	
\hspace{1cm} ในทำนองเดียวกันกับตัวแบบการต่อเติมภาพเฉดสีเทาที่ใช้การแปรผันรวม ตัวแบบการต่อเติมภาพสีที่ใช้การแปรผันรวมสามารถเขียนได้ดังนี้
\begin{align}
	\min_{\boldsymbol{u}} \{ \bar{\mathcal{J}}(\boldsymbol{u})= \mathcal{\bar{D}}(\boldsymbol{u},\boldsymbol{z})+  \mathcal{\bar{R}}(\boldsymbol{u}) \}
	\label{e10}
\end{align}
เมื่อ
\begin{align*}
	\mathcal{\bar{D}}(\boldsymbol{u},\boldsymbol{z}) 
	&= \frac{1}{2}\int_{\Omega}^{}\lambda(u_1 - z_1)^2 d\Omega + \frac{1}{2}\int_{\Omega}^{}\lambda(u_2 - z_2)^2 d\Omega + \frac{1}{2}\int_{\Omega}^{}\lambda(u_3 - z_3)^2 d\Omega
\end{align*}
และ 
\begin{align*}
	\mathcal{\bar{R}}(\boldsymbol{u})= \int_{\Omega}^{}\lvert\nabla u_1 \rvert d\Omega + \int_{\Omega}^{}\lvert\nabla u_2 \rvert d\Omega + \int_{\Omega}^{}\lvert\nabla u_3 \rvert d\Omega
\end{align*}
	
\hspace{1cm}  หลังจากใช้วิธีสปิทเบรกแมนกับ (\ref{e10}) จะได้
\begin{align}
	\min_{\boldsymbol{u},\boldsymbol{w}_1,\boldsymbol{w}_2,\boldsymbol{w}_3} \{\bar{\mathcal{J}}(\boldsymbol{u},\boldsymbol{w}_1,\boldsymbol{w}_2,\boldsymbol{w}_3)&= \mathcal{\bar{D}}(\boldsymbol{u},\boldsymbol{z}) +  \underset{l=1}{\overset{3}{\sum}} \int_{\Omega}^{}|\boldsymbol{w}_l|d\Omega
	\nonumber\\
	&\quad+ \frac{\theta_l}{2} \underset{l=1}{\overset{3}{\sum}}\int_{\Omega}^{}(\boldsymbol{w}_l - \nabla u_l - \boldsymbol{b_l})^{2}d\Omega\}, \hspace{1cm} \theta_l > 0
\end{align}
	
\vspace{1cm}
\hspace{1cm}ขั้นตอนวิธีสปริทเบรกแมนสำหรับภาพสี แสดงได้ดังนี้ 
\vspace{0.5cm}
\begin{algorithm}[H]
    \SetAlgoNoLine
    \caption{SB Method for Color TV-based Image Inpainting}
    \SetKwFunction{FMain}{$\boldsymbol{u} \longleftarrow SBC$}
    \SetKwProg{Fn}{}{}{}
    \Fn{\FMain{$\boldsymbol{u}, \boldsymbol{z} \lambda, \theta, N_{gs}, N, \varepsilon$}}{
        \For{$l = 1:3$}{
        $u_l \longleftarrow SB(u_l, z_l \lambda, \theta, N_{gs}, N, \varepsilon) $
        }
    }
\end{algorithm}
\vspace{1cm}

