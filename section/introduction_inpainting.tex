\section{บทนำของการต่อเติมภาพ}

\hspace{1cm} ในปัจจุบันการใช้ภาพดิจิตัล (digital images) ในสังคมเครือข่ายได้รับความนิยมอย่างแพร่หลาย เนื่องจากโทรศัพท์เคลื่อนที่มีราคาถูกลงแต่มีความสามารถที่ชาญฉลาด สามารถทำหน้าที่ได้ตั้งแต่การเป็นกล้องดิจิตัลคอมแพค (compact digital camera)  คุณภาพดีให้ภาพดิจิตัลที่มีความคมชัดสูงจนไปถึงการทำหน้าที่ดังเช่นเครื่องคอมพิวเตอร์ส่วนบุคคลที่สามารถเชื่อมต่อกับระบบเครือข่ายไร้สายเพื่อรับส่งภาพดิจิตัลในสังคมเครือข่ายด้วยความสะดวกและรวดเร็ว

\hspace{1cm} นอกจากภาพดิจิตัลจะได้รับจากการถ่ายภาพด้วยโทรศัพท์เคลื่อนที่แล้ว ภาพดิจิตัลยังได้รับการถ่ายภาพด้วยกล้องดีเอสแอลอาร์ หรือ กล้องสะท้อนเลนส์เดี่ยวแบบดิจิตัล (digital single lens reflex camera) กล้องโทรทรรศน์ (หรือ กล้องดูดาว) หรือ เครื่องมือสร้างภาพถ่ายทางการแพทย์ (medical imaging device) 

\hspace{1cm} โดยทั่วไปภาพดิจิตัลจะได้รับการประมวลผลภาพก่อนนำไปใช้งานเพื่อให้สามารถใช้ข้อมูลที่ปรากฎบนภาพได้ตรงวัตถุประสงค์ของการใช้งานมากที่สุด ตัวอย่างเช่น ภาพบุคคล (portrait) อาจจำเป็นต้องได้รับการกำจัดสัญญาณรบกวนออกจากภาพและ/หรือปรับเพิ่มความละเอียดข้อมูลของความเข้มของสีและความสว่างของสีบนบริเวณใบหน้าก่อนนำภาพไปใช้งานเพื่อจัดทำต้นฉบับวารสารหรือหนังสือของสำนักพิมพ์ เป็นต้น  

\hspace{1cm} การต่อเติมภาพ (image inpainting) เป็นวิธีการประมวลผลภาพชนิดหนึ่งมีเป้าหมายเพื่อซ่อมแซมภาพด้วยการต่อเติมข้อมูลของความเข้มของสีบนบริเวณที่กำหนด (ต่อไปจะเรียกบริเวณนี้ว่าโดเมนต่อเติม (inpainting domain)) โดยอาศัยข้อมูลของความเข้มของสีที่ปรากฏในภาพ ตัวอย่างเช่น 
กำหนดให้รูปที่ \ref{figure:inpaint-explain} (\subref{figure:inpaint-explain:to-inpaint}) แสดงภาพที่ต้องการซ่อมแซมระดับความเข้มของสีบนบริเวณแท่งวัตถุรูปร่างสี่เหลี่ยมสีขาว การต่อเติมภาพดังกล่าวจะเริ่มด้วยการกำหนดให้บริเวณแท่งวัตถุรูปร่างสี่เหลี่ยมสีขาวเป็นโดเมนการต่อเติมดังรูปที่ \ref{figure:inpaint-explain} (\subref{figure:inpaint-explain:inpaint-domain})  จากนั้นภาพที่ได้รับการซ่อมแซมหรือภาพที่ได้รับการต่อเติม (restored or inpainted image) ซึ่งแสดงในรูปที่ \ref{figure:inpaint-explain} (\subref{figure:inpaint-explain:inpainted}) ได้มาจากขั้นตอนวิธีการต่อเติมภาพ (inpainting algorithm) ซึ่งได้รับการออกแบบเพื่อนำข้อมูลที่ปรากฎบนภาพในบริเวณใกล้เคียงกับขอบของโดเมนต่อเติมมาซ่อมแซมภาพ 
	
\begin{figure}[H]
	\centering
	\begin{subfigure}{0.3\linewidth}
		\centering
		\includegraphics[width=0.8\linewidth]{image/grayscale_inpaint/toinpaint.png}
        \caption{ภาพที่ต้องการซ่อมแซม}
        \label{figure:inpaint-explain:to-inpaint}
	\end{subfigure}
	\begin{subfigure}{0.3\linewidth}
		\centering
		\includegraphics[width=0.8\linewidth]{image/grayscale_inpaint/inpaintdomain.png}
        \caption{โดเมนต่อเติม}
        \label{figure:inpaint-explain:inpaint-domain}
	\end{subfigure}
	\begin{subfigure}{0.3\linewidth}
		\centering
		\includegraphics[width=0.8\linewidth]{image/grayscale_inpaint/result_splitbergman.png}
        \caption{ภาพที่ได้รับการซ่อมแซม}
        \label{figure:inpaint-explain:inpainted}
	\end{subfigure}
	\caption{ตัวอย่างการซ่อมแซมภาพ}
	\label{figure:inpaint-explain}
\end{figure}

\hspace{1cm} เท่าที่ผู้วิจัยศึกษาและค้นคว้ามาจนถึงขณะนี้ ผู้วิจัยพบว่าการต่อเติมภาพมักนิยมนำไปใช้งานสำหรับการปรับแต่งความสวยงามของภาพบุคคลที่ถ่ายจากโทรศัพท์เคลื่อนที่ เช่น การลบร่องรอยของรอยตีนกา การลบร่องรอยแผลเป็นที่เกิดจากสิวเสี้ยน การลดร่องรอยของความชรา หรือ การเพิ่มความใสและความเนียนของสีผิวบนบริเวณใบหน้าผ่านโปรแกรมแอปพลิเคชันแต่งรูปภาพที่มีอยู่ในแอปสโตร์ (App Store) หรือ กูเกิ้ลเพลย์ (Google Play) เป็นต้น