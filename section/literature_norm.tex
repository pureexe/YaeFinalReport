\section{ปริภูมิที่มีค่าประจำ}

\hspace{1cm} ค่าประจำ (norm) เป็นเครื่องมือที่ใช้สำหรับบอกขนาดของเวคเตอร์หนึ่งในปริภูิมเวคเตอร์ ซึ่งค่าประจำนั้น มีนิยามดังนี้

\begin{Definition}
    (ค่าประจำ)  ค่าประจำบนปริภูมิเวคเตอร์ $V$ คือฟังก์ชันค่าจริง $ || \cdot ||$ ซึ่งนิยามบน $V$ โดยที่ 
    \begin{enumerate}
        \item $ ||u|| > 0 $ ถ้า $ u \neq 0 $
        \item $|| \lambda u || = | \lambda | || u || $ สำหรับทุกสเกลาร์ $\lambda$ และทุกเวคเตอร์ $u$
        \item $ ||u+v|| \leq ||u|| + ||v || $ สำหรับทุก $u,v \in V$
    \end{enumerate}
\end{Definition}

\hspace{1cm} โครงงานวิจัยเรื่องนี้รูปภาพเฉดเทาเป็นฟังก์ชันที่อยู่ในปริภูมิแบบยูคลิด (Euclidean space) นั่นคือค่าประจำทั้งหมดที่พูดถึงในโครงงานวิจัยเรื่องนี้จะเป็นค่าประจำแบบยูคลิด ซึ่งมีนิยามดังนี้

\begin{Definition} (ค่าประจำแบบยูคลิด)
    ปริภูมิยูคลิด n มิติ สามารถเขียนเวคเตอร์ในปริภูมิยูคลิดได้ว่า $ \boldsymbol{x} = (x_1,x_2, ... , x_n)$
    \begin{align*}
        || \boldsymbol{x} || = \sqrt{x_1^2 + x_2^2 + ... + x_n^2 } 
    \end{align*}
\end{Definition}