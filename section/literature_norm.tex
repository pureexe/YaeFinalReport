\section{ปริภูมิที่มีค่าประจำ}

%สำหรับโครงงานวิจัยเรื่องนี้ จะสามารถพิจารณาฟังก์ชันรูปภาพเป็นค่าจำนวนจริงได้โดยการใช้ ค่าประจำ (Norm)

%โดยฟังก์ชันภาพจะสามารถพิจารณาเป็นเมทริกซ์ของ

\hspace{1cm} ค่าประจำ (Norm) คือฟังก์ชันที่มีค่ามากกว่าเท่ากับศูนย์ โดยจะเรียก $p$ ว่าเป็นฟังก์ชันค่าประจำเมื่อ \\$p : V \rightarrow [0,+\infty)$ สำหรับ ทุก $a \in F$  และทุก $u, v \in V$ เมื่อ $V$ เป็นปริภูมิเวกเตอร์ และ $F$ เป็นฟิลด์ โดยที่มีสมบัติดังต่อไปนี้
\begin{enumerate}
    \item $p(u + v) \leq p(u) + p(v)$
    \item $p(av) = |a| p(v)$ 
    \item ถ้า $p(v) = 0$ แล้ว $v=0$ 
\end{enumerate}

\hspace{1cm} สำหรับโครงงานวิจัยนี้รูปภาพเฉดเทาจะเขียนเป็นฟังก์ชันที่อยู่ในปริภูมิยูคลิด (Eucliden space) หรือ $\mathbb{R}^2$ จึงทำให้ค่าประจำทั้งหมดที่พูดถึงในโครงงานวิจัยนี้จะใช้ Norm $L_2$ หรือระยะทางแบบยูคลิด (Euclidean distance) ทั้งหมด ซึ่งค่าประจำนี้จะกำหนดโดย

\begin{align*}
    || X || = \sqrt{x_1 + x_2 + ... + x_n} \hspace{1cm} \forall x_i \in X
\end{align*}