\section{ปริภูมิที่มีค่าประจำ}

\begin{Definition}
    (ค่าประจำ)  ค่าประจำบนปริภูมิเวกเตอร์เชิงเส้น $V$ คือฟังก์ชันค่าจริง $ || \cdot ||$ ซึ่งนิยามบน $V$ และสอดคล้องสมบัติต่อไปนี้ 
    \begin{enumerate}
        \item $ ||u|| > 0 $ เมื่อ $ u \neq 0 \in V $
        \item $|| \lambda u || = | \lambda | || u || $ สำหรับทุกสเกลาร์ $\lambda$ และทุกเวกเตอร์ $u$
        \item $ ||u+v|| \leq ||u|| + ||v || $ สำหรับทุก $u,v \in V$
    \end{enumerate}
\end{Definition}

\begin{Definition} ปริภูมิที่มีค่าประจำ (Norm space) คือปริภูิมเวกเตอร์เชิงเส้น $V$ ซึ่งมีค่าประจำ $ || \cdot ||$
\end{Definition}

\begin{Example}
    ถ้า $V = \mathbb{R}^{n}$ ได้ว่า $V$ เป็นปริภูมิเวกเตอร์เชิงเส้น ซึ่งค่าประจำ $ || \cdot ||$ สามารถนิยามได้โดย
    \begin{align}
        || \boldsymbol{x} || = \sqrt{x_1^2 + x_2^2 + ... + x_n^2 } 
    \end{align}
    และค่าประจำนี้เรียกว่า ค่าประจำแบบยูคลิด (Euclidean norm) 
\end{Example}

\textbf{หมายเหตุ:} $ || \cdot ||$ สัญลักษณ์นี้ต่อไปในเอกสารนี้จะใช้แทนค่าประจำ