\section{แคลคูลัสของการแปรผันเบื้องต้น}

\hspace{1cm} แคลคูลัสของการแปรผัน (Calculus of variations) คือสาขาวิชาในวิชา คณิตศาสตร์วิเคราห์ เพื่อใช้สำหรับแก้ปัญหาค่าเหมาะสม โดยจะสนใจที่จะหาฟังก์ชันที่เหมาะสมแทนที่จะหาค่าของตัวแปรที่เหมาะสม แคลคูลัสของการแปรผันนั้นมักจะเกี่ยวข้องกับปัญหาที่ต้องการปริมาณน้อยสุดหรือมากสุดซึ่งปรากฏอยู่ในรูปของอนุพันธ์หรือปริพันธ์ที่ไม่ทราบค่าฟังก์ชัน

\hspace{1cm}ซึ่งฟังก์ชันหาค่าต่ำสุด มักจะมีรูปทั่วไปดังสมการ  \ref{equation:general-minimize} 

\begin{align}
    \underset{u}{{min}} \int \mathcal{J}(u)
    \label{equation:general-minimize}
\end{align}

โดยที่ $\mathcal{J} : \mathbb{U} \rightarrow \mathbb{R} $ เป็นฟังก์ชันที่ส่งจากฟังก์ชันไปยังจำนวนจริง เรียกว่า ฟังก์ชันนัล (functional) และกำหนด $\mathbb{U}$ เป็นปริภูมิของคำตอบซึ่งประกอบด้วยฟังก์ชันค่าต่ำสุดของ $\mathcal{J}$ 

