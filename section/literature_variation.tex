\section{แคลคูลัสของการแปรผันเบื้องต้น}

\hspace{1cm} แคลคูลัสของการแปรผัน (Calculus of variations) คือสาขาวิชาในวิชา คณิตศาสตร์วิเคราห์ เพื่อใช้สำหรับแก้ปัญหาค่าเหมาะสม โดยจะสนใจที่จะหาฟังก์ชันที่เหมาะสมแทนที่จะหาค่าของตัวแปรที่เหมาะสม โดยการแก้ปัญหาเหล่านี้จะอยู่ในรูปแบบของฟังก์ชัน (functional) ซึ่งฟังก์ชันเหล่านี้เองจะอยู่ในรูปแบบของปริพันธ์หรืออนุพันธ์ ซึ่งเราสามารถใช้วิธีการแปรผัน (varational methods) เพื่อแปลงปัญหาดังกล่าวให้อยู่ในรูปแบบของสมการอนุพันธ์ย่อย (partial differential equation : PDE) ได้ โดยฟังก์ชันที่ได้จากการแก้สมการเชิงอนุพันธ์นั้นจะถูกแปรผันไปตามคำตอบของสมการอนุพันธ์เชิงย่อย จึงทำให้เรียกถูกเรียกว่าการแปรผัน

\hspace{1cm}ซึ่งฟังก์ชันหาค่าต่ำสุด มักจะมีรูปทั่วไปดังสมการ  \ref{equation:general-minimize} โดยที่ $\mathbb{U}$ เป็นปริภูมิของคำตอบซึ่งประกอบด้วยฟังก์ชันค่าต่ำสุดของ $\mathcal{J}$ และกำหนด $\mathcal{J} : \mathbb{U} \rightarrow  \mathbb{R} $

\begin{align}
    \underset{u}{{min}} \int \mathcal{J}(u)
    \label{equation:general-minimize}
\end{align}

\hspace{1cm} ซึ่งปัญหาการหาค่าต่ำสุดนี้สามารถเขียนในรูปแบบของ สมการอนุพันธ์ย่อยได้ ซึ่งสมการอนุพันธ์ย่อยนี้มีอีกชื่อหนึ่งว่า สมการ ออยเลอร์-ลากรางซ์ โดยใช้วิธีการแปรผัน เพื่อหาค่าต่ำสุดได้ดังนี้

\begin{align*}
    \frac{\partial }{\partial u } \mathcal{J}(u) = 0
\end{align*}