\section{แคลคูลัสของการแปรผันเบื้องต้น}

% สั้นไป
% energy unction min int J(u)
% พูดถึง minimizer
% ลงไปให้ถึงการดริฟ
% หาหนังสือ functional analysis / Optimization
% จะบอคำตอบของ variational form ได้จาก pde โดยกราฟบอกว่ามันเชื่อมไปลากราซ์อย่างไร อย่าลืมยกตัวอย่าง


\hspace{1cm} แคลคูลัสของการแปรผัน (Calculus of variations) คือสาขาวิชาในวิชา คณิตศาสตร์วิเคราห์ เพื่อใช้สำหรับแก้ปัญหาค่าเหมาะสม โดยจะสนใจที่จะหาฟังก์ชันที่เหมาะสมแทนที่จะหาค่าของตัวแปรที่เหมาะสม แคลคูลัสของการแปรผันนั้นมักจะเกี่ยวข้องกับปัญหาที่ต้องการปริมาณน้อยสุดหรือมากสุดซึ่งปรากฏอยู่ในรูปของอนุพันธ์หรือปริพันธ์ที่ไม่ทราบค่าฟังก์ชัน

\hspace{1cm}ซึ่งฟังก์ชันหาค่าต่ำสุด มักจะมีรูปทั่วไปดังสมการ  \ref{equation:general-minimize} 
\begin{align}
    \underset{u}{{min}} \int \mathcal{J}(u)
    \label{equation:general-minimize}
\end{align}

โดยที่ $\mathcal{J} : \mathcal{U} \rightarrow \mathbb{R} $ เป็นฟังก์ชันที่ส่งจากฟังก์ชันไปยังจำนวนจริง เรียกว่า ฟังก์ชันนัล (functional) พร้อมทั้งกำหนด $\mathcal{U}$ เป็นปริภูมิของคำตอบซึ่งประกอบด้วยฟังก์ชันค่าต่ำสุดของ $\mathcal{J}$  และ $\mathcal{V}$ เป็นปริภูมิทดสอบซึ่งบริภูมินี้สามารถเขียนเป็นผลต่างระหว่าง 2 ฟังก์ชันได้ นั่นคือ
\begin{align}
    \mathcal{V} = \{v|v = u - \hat{u} \hspace{0.25cm} \text{และ} \hspace{0.25cm} u,\hat{u} \in \mathcal{U} \}
\end{align}

\begin{Definition}
    (ย่านใกล้เคียง) ให้ $\mathcal{U}$ เป็นปริภูมิคำตอบ ฟังก์ชัน $\hat{u} \in \mathcal{U}$ และ $\epsilon > 0$ แล้ว $\mathcal{B}_{\epsilon}$ จะเป็นย่านใกล้เคียงของ $\hat{u}$ เมื่อ
    \begin{align*}
        \mathcal{B}_{\epsilon} = \{ u \in \mathcal{U} | || u - \hat{u} || < \epsilon \}
    \end{align*}
\end{Definition}

จาก \ref{equation:general-minimize} โลคอลมินิไมเซอร์ (local minimizer) จะกำหนดโดย 

\begin{Definition}
    (โลคอลมินิไมเซอร์) ให้ $\mathcal{U}$ เป็นปริภูมิคำตอบ และฟังก์ชันนัล $\mathcal{J} : \mathcal{U} \rightarrow \mathbb{R} $ จะเรียก $\hat{u} \in \mathcal{U}$ ว่าโลตอลมินิไมเซอร์ของ $\mathcal{J}$ ถ้าทุก $\epsilon > 0 $ มี $\delta > 0$ โดยที่ $\mathcal{J}(\hat{u}) \leq \mathcal{J}(u)$ สำหรับทุก $u \in \mathcal{B}_{\epsilon}(\hat{u})$ 
\end{Definition}

ในการนิยามเงื่อนไขที่จำเป็นสำหรับโลคอลมินิไมเซอร์ของ $\mathcal{J}$ จำเป็นต้องมีการกำหนดการหาอนุพันธ์แบบมีทิศทาง

\begin{Definition}
    (G\^{a}teaus-differentiable)
    $\mathcal{U}$ เป็นปริภูมิคำตอบม $\mathcal{V}$ เป็นปริภูมิทดสอบ และฟังก์ชันนัล $\mathcal{J} : \mathcal{U} \rightarrow \mathbb{R}$ แล้ว $\mathcal{J}$ เป็น G\^{a}teaus-differentiable เมื่อทุก $u \in \mathcal{U}$ อยู่ในทิศทางของ $v \in \mathcal{V}$ ถ้า
    \begin{enumerate}
        \item มีจำนวน $\hat{\epsilon} > 0$ ซึ่งทำให้ $u_{\epsilon} = u + \epsilon v \in \mathcal{U}$ สำหรับทุก $|e| \leq \hat{\epsilon}$
        \item ฟังก์ชัน $J(\epsilon) = \mathcal{J}(u_\epsilon)$
    \end{enumerate}
    โดยอนุพันธ์อันดับแรกของ G\^{a}teaus หรือการแปรผันอันดับแรก (first variation) ของ $\mathcal{J}$ สำหรับ $u$ ที่อยู่ในทิศทางของ $v$ กำหนดโดย
    \begin{align*}
        \delta \mathcal{J}(u;v) = J'(0) = \frac{d\mathcal{J}(u + \epsilon v)}{d \epsilon} \Big|_{\epsilon = 0} = \lim_{\epsilon = 0}\frac{\mathcal{J}(u + \epsilon v)}{\epsilon} 
    \end{align*}
\end{Definition}

และนิยามจุดคงตัว (stationary point) โดย

\begin{Definition}
(จุดคงตัว) $\mathcal{U}$ เป็นปริภูมิคำตอบม $\mathcal{V}$ เป็นปริภูมิทดสอบ และฟังก์ชันนัล $\mathcal{J} : \mathcal{U} \rightarrow \mathbb{R}$ สมมติให้ บาง $\hat{u} \in \mathcal{U}$ แล้ว $\mathcal{J}$ เป็น G\^{a}teaus-differentiable สำหรับทุกฟังก์ชันทดสอบ $v \in \mathcal{V}$ แล้ว $\hat{u}$ จะเรียกว่าจุดคงของ $\mathcal{J}$ ก็ต่อเมื่อ $\delta\mathcal{J}(\hat{u};v) = 0$ สำหรับทุก $v \in \mathcal{V}$
\end{Definition}

เงื่อนไขที่จำเป็นสำหรับมินิไมเซอร์นั้นสามารถสร้างได้จากการใช้จุดคงตัว

\begin{Theorem}
    (เงื่อนไขที่จำเป็นสำหรับโลคอลมินิไมเซอร์) ให้ $\mathcal{U}$ เป็นปริภูมิคำตอบ ซึ่ง $\hat{u} \in \mathcal{U}$, ฟังก์ชันนัล $\mathcal{J} : \mathcal{U} \rightarrow \mathbb{R}$, $\mathcal{V}$ เป็นปริภูมิทดสอบโดยที่ $v$ เป็นฟังก์ชันทดสอบซึ่ง $v \in \mathcal{V}$ และ $\mathcal{J}$ เป็น G\^{a}teaus-differentiable สำหรับทุก $\hat{u}$\\
    ถ้า $\hat{u}$ เป็นโลคอลมินิไมเซอร์ของ $\mathcal{J}$ แล้ว $\hat{u}$ เป็นจุดคงตัวของ  $\mathcal{J}$\\
    สำหรับบทพิสูจน์จะพบได้ใน \cite{ref:calculus_of_variation}
    \label{theroem:nessery_minimizer}
\end{Theorem}

ด้วยทฤษฎีบทนี้เราจะศึกษาเงื่อนไขของจุดคงตัวเพิ่มเติมโดยเราเลือกฟังก์ชันนัลทั่วไป $\mathcal{J}$ ซึ่งนิยามโดย
\begin{align}
    \mathcal{J}(u) = \int_{\Omega} F[x,u(x),\nabla u(x)] dx
    \label{equation:general_functional}
\end{align}

โดยที่ $\Omega \subset \mathbb{R}^{d}, d>1$ เป็นเซ็ตเปิดมีขอบเขตและ $F$ เป็นฟังก์ชันนัลที่ขึ้นอยู่กับ $x=(x_1,x_2,...,X_d)^\top$ สมมติให้ $\mathcal{J}$ เป็น G\^{a}teaus-differentiable ในทุกทิศทางของปริภูมิทดสอบ ดังนั้นจึงสมมุติได้ว่า $F$  เป็นอนุพันธ์ย่อยที่ต่อเนื่อง

ก่อนที่จะแนะนำเงื่อนไขสำหรับจุดคงตัวของ $\mathcal{J}$ จะแนะนำ
\begin{align}
    \nabla_{u} F = \partial F / \partial u = F_u
\end{align}

สำหรับเกรเดียนซ์ของ $F$ ซึ่งขึ้นอยู่กับ $u$ จะกำหนดโดย
\begin{align}
    \nabla F = (\partial F / \partial x_1, ... \partial F / \partial x_d, )^\top
\end{align}

ในทำนองเดียวกันเกรเดียนซ์ของ $F$ ซึ่งขึ้นอยู่กับ $\nabla u$ กำหนดโดย
\begin{align}
    \nabla_{\nabla u} F = (\partial F / \partial u_{x_1},..., \partial F / \partial u_{x_d})^\top \in \mathbb{R}^d
\end{align}

โดยในขั้นนี้เราจะเลือกคำตอบที่เจาะจงโดยการเพิ่มเงื่อนไขค่าขอบเข้าไป ตัวอย่างเช่น
\begin{align}
    \tilde{\mathcal{U}} = \{u \in \mathcal{U} | u = c \text{ บน } \partial \Omega \}
\end{align}

และเช่นเดียวกันปริภูมิทดสอบจะถูกกำหนดโดย
\begin{align}
    \tilde{\mathcal{V}} = \{v \in \mathcal{V} | v = 0 \text{ บน } \partial \Omega \}
\end{align}

อีกทั้งขั้นตอนที่กล่าวมาไม่เพียงครอบคลุมปริภูมิทั่วไป $\mathcal{U}$ และ $\mathcal{V}$ แต่ยังครอบคลุมไปถึงกรณีที่เป็นเวคเตอร์เมื่อ $ \boldsymbol{u} = (u_1,u_2,...,u_d)^\top : \mathbb{R}^{d} \rightarrow \mathbb{R}^{d}$

\begin{Lemma}
    (จุดคงตัวของ $\mathcal{J}$) ฟังก์ชัน $u \in \mathcal{U}$ เป็นจุดคงตัวของฟังก์ชันนัลทั่วไป $\mathcal{J}$ \ref{equation:general_functional} ถ้า
    \begin{align}
        \int_{\Omega} \Big \langle \nabla_u F - \nabla \cdot \nabla_{\nabla u} F,v \Big \rangle_{\mathbb{R}^{d}} dx = 0
        \label{equation:lemma_stationarypoint}
    \end{align}
    \label{lemma:stationary_point}
    บทพิสูจน์ของทฤษฏีบทประกอบนี้สามารถดูได้จาก \cite{ref:general_functional_stationary}
\end{Lemma}

เห็นได้ชัดว่า \ref{equation:lemma_stationarypoint} เป็นจริงสำหรับฟังก์ชันทดสอบคงตัวที่ค่าไม่เจาะจง เพราะถ้า $\nabla_u F - \nabla \cdot \nabla_{\nabla u}$ การยืนยันทฤษฏีบทนี้ เป็นทฤษฏีบทที่รู้กันดีอยู่แล้ว ดังนั้น $u \in \tilde{U}$ เป็นจุดคงตัวของฟังก์ชันนัล $\mathcal{J}$ ซึ่งเป็น G\^{a}teaus-differentiable ถ้า 
\begin{align}
    \nabla u F - \nabla \ \nabla_{\nabla u} F = 0 \text{ บน } \Omega
    \label{equation:lemma_stationarypoint2}
\end{align}

โดยการใช้ทฤษฏีบท \ref{theroem:nessery_minimizer} กับ  \ref{equation:lemma_stationarypoint2} จะได้เงื่อนไขที่จำเป็นสำหรับโลคอลมินิไมเซอร์ของ \ref{equation:general-minimize} จาก $ d > 1$ และ \ref{equation:lemma_stationarypoint2} จะนำไปสู่สมการเชิงอนุพันธ์ย่อย ซึ่งรู้จักกันในชื่อของสมการออยเลอร์-ลากรางซ์ ที่มีเงื่อนไขค่าขอบ โดยจะเรียก \ref{equation:general-minimize} ที่มีเงื่อนไขค่าขอบว่า รูปแบบการแปรผัน (variational formulation) และถ้าเงื่อนไขค่าขอบนั้นถูกกำหนดไว้ชัดเจนตามทฤษฏีบทประกอบที่ \ref{lemma:stationary_point} จะเรียกเงื่อนไขนี้ว่า เงื่อนไขจำเป็น (essential condition) และในทางกลับกันหากค่าขอบไม่ถูกกำหนดไว้ชัดเจนจะเรียกว่า เงื่อนไขธรรมชาติ (natural condition) 

โดยสรุปแล้วทุกคำตอบ $u* \in \mathcal{U}$ ในปัญหาค่าเหมาะสมทั่วไปดังเช่น \ref{equation:general-minimize} ที่มีฟังก์ชันนัล $\mathcal{J}$ ซึ่งเป็น G\^{a}teaus-differentiable ที่ถูกกำหนดโดย \ref{equation:general_functional} จะเป็นคำตอบของปัญหาค่าขอบซึ่งเป็นส่วนประกอบของสมการออยเลอร์-ลากรางซ์
\begin{align*}
    \nabla_u F - \nabla \cdot \nabla_{\nabla u} F = 0 \text{ บน } \Omega
\end{align*}

ส่วนเงื่อนไขค่าขอบนั้นสามารถเป็นแบบเงื่อนไขจำเป็น หรือเงื่อนไขธรรมชาติก็ได้
\begin{align*}
    \langle \nabla_{\nabla u} F, n \rangle_{\mathbb{R}^{d}} = 0 \text{ บน } \partial \Omega
\end{align*}

เมื่อ $n = (n_1, ..., n_d)^\top$ หมายถึงเวคเตอร์หนึ่งหน่วยปกติภายนอก (outer normal vector unit) ของ $\partial \Omega$

\begin{Example}
    ให้ $d = 2, \Omega = [0,1]^2, F = |\nabla u| \text{เมื่อ} u = u(x)$ และรูปแบบแปรผันเป็น 
    \begin{align*}
        \underset{min}{u} \int_{\Omega} |\nabla u| dx 
    \end{align*}
    กำหนดให้ \[
        \mathcal{R}(u) = \int_\Omega |\nabla u| d \Omega = \int_\Omega \sqrt{u_x^2+u_y^2} d \Omega
    \]
    สำหรับการแปรผันอันดับหนึ่งของ  $\mathcal{R}$ กำหนดให้ $\Phi (s)=s$ จะได้
    \begin{align*}
    \frac{\delta}{\delta u}\mathcal{R}(u;v) &= \left.\frac{d}{d\varepsilon}\mathcal{R}(u+\varepsilon v) \right|_{\varepsilon = 0} 
    = \left.\frac{d}{d\varepsilon} \int_{\Omega} \Phi (|\nabla (u + \varepsilon v)|)d\Omega 
    \right|_{\varepsilon = 0} 
    \end{align*}
    ดังนั้น
    \begin{align*}
    \frac{\delta}{\delta u}\mathcal{R}(u;v) 
    &= \int_{\Omega} \left.\frac{d}{d\varepsilon} \Phi ({\sqrt{(u_x+\varepsilon v_x)^2 + (u_y+\varepsilon v_y)^2 }})\right|_{\varepsilon = 0} d\Omega \\
    &= \int_{\Omega}^{}\left[\Phi'({\sqrt{(u_x+\varepsilon v_x)^2 + (u_y+\varepsilon v_y)^2 }}) \frac{(u_x+\varepsilon v_x)v_x}{{\sqrt{(u_x+\varepsilon v_x)^2+ (u_y+\varepsilon v_y)^2 }}}\right.\\
    &\quad + \Phi'({\sqrt{(u_x+\varepsilon v_x)^2 + (u_y+\varepsilon v_y)^2 }}) \left.\left.\frac{(u_y+\varepsilon v_y)v_y}{{\sqrt{(u_x+\varepsilon v_x)^2 + (u_y+\varepsilon v_y)^2 }}}\right] \right|_{\varepsilon = 0} d\Omega 
    \\
    &= \int_{\Omega} \Phi'(|\nabla u|)\left(\frac{u_x v_x}{|\nabla u|} + \frac{u_y v_y}{|\nabla u|} \right) d\Omega \\
    &= \int_{\Omega} \frac{\Phi'(|\nabla u|)}{|\nabla u|} (\nabla u \cdot \nabla v)d\Omega \\
    &= \int_{\Omega}\Phi'(|\nabla u|) \frac{\nabla u}{|\nabla u|} \cdot \nabla v d\Omega
    \end{align*}
    โดยเอกลักษณ์อันดับหนึ่งของกรีน จะได้ว่า
    \begin{align*}
    \frac{\delta}{\delta u}\mathcal{R}(u;v) &= - \int_{\Omega}^{} v \nabla \cdot \left(\Phi'(|\nabla u|) \frac{\nabla u}{|\nabla u|} \right)d\Omega + \int_{\partial \Omega}^{} v \left(\Phi'(|\nabla u|) \frac{\nabla u}{|\nabla u|} \cdot \boldsymbol{n}\right) dS \\
    &= -\int_{\Omega}^{} v \nabla \cdot \left(\Phi'(|\nabla u|) \frac{\nabla u}{|\nabla u|} \right)d\Omega + \int_{\partial \Omega}^{} v \left(\Phi'(|\nabla u|)  \frac{\partial u}{\partial \boldsymbol{n}}\right) dS
    \end{align*}
    เพราะฉะนั้น  
    \[
    \frac{\delta}{\delta u}\mathcal{R}(u;v) = -\int_{\Omega}^{} v \nabla \cdot \left(\Phi'(|\nabla u|) \frac{\nabla u}{|\nabla u|} \right)d\Omega + \int_{\partial \Omega}^{} v \left(\Phi'(|\nabla u|)  \frac{\partial u}{\partial \boldsymbol{n}}\right) dS
    \]
    เมื่อ $  \boldsymbol{n} $ แทนเวกเตอร์หนึ่งหน่วยที่ตั้งฉากกับขอบของภาพในทิศทางชี้ออก

    จะได้สมการออยเลอร์ที่สมนัยกับปัญหานี้คือ
    \begin{align*}
        - \triangle u = 0 \text{ บน } \Omega \\
        \frac{ \partial  u }{ \partial \boldsymbol{n}} = 0 \text{ บน } \partial \Omega
    \end{align*}
\end{Example}