\section{แคลคูลัสของการแปรผันเบื้องต้น}

\hspace{1cm} ในหัวข้อย่อยนี้ เราจะกล่าวถึงปัญหาการหาค่าเหมาะสมที่สุดประเภทหนึ่งสำหรับหาฟังก์ชันที่เหมาะสม (แทนที่จะเป็นการหาค่าเหมาะสมของตัวแปรดังเช่นปัญหาการหาค่าเหมาะสมที่สุดทั่วไป) ซึ่งทำให้ปริมมาณที่กำหนด (มักนำเสนอปริมาณนี้ในรูปของอินทิกรัลจำกัดเขต) มีภาวะหยุดนิ่ง (stationary) เนื่องจากค่าของฟังก์ชันเปลี่ยนแปลงตามตัวแปรอิสระ ปัญหาการหาค่าเหมาะสมที่สุดประเภทนี้จึงถูกแก้ด้วยแคลคูลัสของการแปรผันทั้งนี้แคลคูลัสของการแปรผันสามารถนำมาใช้หาค่าขีดสุดของปริมาณที่เขียนในรูปอินทิกรัลจำกัดเขตที่ประกอบด้วยฟังก์ชันไม่ทราบค่าและ/หรืออนุพันธ์ของฟังก์ชันไม่ทราบค่า

\hspace{1cm} พิจารณาปัญหาค่าต่ำที่สุดต่อไปนี้
\begin{align}
    \underset{u}{\min} \mathcal{J}(u)
    \label{equation:general-minimize}
\end{align}

โดยที่ $\mathcal{J} : \mathcal{U} \rightarrow \mathbb{R} $ เป็นฟังก์ชันที่ส่งจากเซตของฟังก์ชันไปยังจำนวนจริง (เราจะเรียก $\mathcal{J}$ ว่า ฟังก์ชันนัล (functional)) $\mathcal{U}$ เป็นปริภูมิของคำตอบซึ่งประกอบด้วยฟังก์ชันที่สามารถทำให้ $\mathcal{J}$ ต่ำที่สุด และ $\mathcal{V}$ เป็นปริภูมิทดสอบซึ่งกำหนดโดย
\begin{align}
    \mathcal{V} = \{v|v = u - \hat{u} \hspace{0.25cm} \text{และ} \hspace{0.25cm} u,\hat{u} \in \mathcal{U} \}
\end{align}

\begin{Definition}
    (ย่านใกล้เคียง) ให้ $\mathcal{U}$ เป็นปริภูมิคำตอบ  $\hat{u} \in \mathcal{U}$ และ $\epsilon > 0$ แล้วย่านใกล้เคียงของ $\hat{u}$ เขียนแทนด้วย $\mathcal{B}_{\epsilon}$ นิยามโดย
    \begin{align*}
        \mathcal{B}_{\epsilon} = \{ u \in \mathcal{U} | || u - \hat{u} || < \epsilon \}
    \end{align*}
\end{Definition}

\begin{Definition}
    (โลคอลมินิไมเซอร์) ให้ $\mathcal{U}$ เป็นปริภูมิคำตอบ และ $\mathcal{J} : \mathcal{U} \rightarrow \mathbb{R} $ เป็นฟังก์ชันนัล จะเรียก $\hat{u} \in \mathcal{U}$ ว่าโลคอลมินิไมเซอร์ของ $\mathcal{J}$ ถ้าสำหรับทุก $\epsilon > 0 $ จะมี $\delta > 0$ ซึ่งทำให้ $\mathcal{J}(\hat{u}) \leq \mathcal{J}(u)$ สำหรับทุกๆ $u \in \mathcal{B}_{\epsilon}(\hat{u})$ 
\end{Definition}

\begin{Definition}
    (G\^{a}teave-differentiable)
    ให้ $\mathcal{U}$ เป็นปริภูมิคำตอบม $\mathcal{V}$ เป็นปริภูมิทดสอบ และ $\mathcal{J} : \mathcal{U} \rightarrow \mathbb{R}$ เป็นฟังก์ชันนัล แล้ว เรากล่าวว่า $\mathcal{J}$ หาอนุพันธ์แบบกาโตวได้สำหรับทุก $u \in \mathcal{U}$ อยู่ในทิศทางของ $v \in \mathcal{V}$ เมื่อ
    \begin{enumerate}
        \item มีจำนวน $\hat{\epsilon} > 0$ ซึ่งทำให้ $u_{\epsilon} = u + \epsilon v \in \mathcal{U}$ สำหรับทุก $|\epsilon| \leq \hat{\epsilon}$
        \item ฟังก์ชัน $J(\epsilon) = \mathcal{J}(u_\epsilon)$ หาอนุพันธ์ได้ที่ $\epsilon = 0$
    \end{enumerate}
    อนุพันธ์กาโตวอันดับหนึ่งหรือการแปรผันอันดับแรก (first variation) ของ $\mathcal{J}$ สำหรับ $u$ ในทิศทางของ $v$ กำหนดโดย
    \begin{align*}
        \delta \mathcal{J}(u;v) = J'(0) = \frac{d\mathcal{J}(u + \epsilon v)}{d \epsilon} \Big|_{\epsilon = 0} = \lim_{\epsilon = 0}\frac{\mathcal{J}(u + \epsilon v)}{\epsilon} 
    \end{align*}
\end{Definition}

\begin{Definition}
(จุดคงตัว) ให้ $\mathcal{U}$ เป็นปริภูมิคำตอบ $\mathcal{V}$ เป็นปริภูมิทดสอบ และ $\mathcal{J} : \mathcal{U} \rightarrow \mathbb{R}$ เป็นฟังก์ชันนัล สมมติให้ $\mathcal{J}$ หาอนุพันธ์แบบกาโตวได้ที่บาง $\hat{u} \in \mathcal{U}$ สำหรับทุกฟังก์ชันทดสอบ $v \in \mathcal{V}$ แล้ว $\hat{u}$ จะเรียกว่าจุดคงตัวของ $\mathcal{J}$ ก็ต่อเมื่อ $\delta\mathcal{J}(\hat{u};v) = 0$ สำหรับทุก $v \in \mathcal{V}$
\end{Definition}

\begin{Theorem}
    ให้ $\mathcal{U}$ เป็นปริภูมิคำตอบ, $\hat{u} \in \mathcal{U}$, $\mathcal{J} : \mathcal{U} \rightarrow \mathbb{R}$ เป็นฟังก์ชันนัล และ $\mathcal{V}$ เป็นปริภูมิทดสอบ สมมติว่า $\mathcal{J}$ หาอนุพันธ์แบบกาโตวได้สำหรับ $\hat{u}$ ในทุกๆ ทิศทางที่ $v \in \mathcal{V}$ ดังนั้น
    \\
    \hspace{1cm} ถ้า $\hat{u}$ เป็นโลคอลมินิไมเซอร์ของ $\mathcal{J}$ แล้ว $\hat{u}$ เป็นจุดคงตัวของ  $\mathcal{J}$
    \label{theroem:nessery_minimizer}
\end{Theorem}
\hspace{1cm} ด้วยทฤษฎีบทนี้เราสามารถสำรวจเงื่อนไขสำหรับจุดคงตัวของฟังก์ชันนัลทั่วไป $\mathcal{J}$ ซึ่งนิยามโดย
\begin{align}
    \mathcal{J}(u) = \int_{\Omega} F[x,u(x),\nabla u(x)] dx
    \label{equation:general_functional}
\end{align}

\hspace{1cm} โดยที่ $\Omega \subset \mathbb{R}^{d} (d>1)$ เป็นเซตเปิดที่มีขอบเขตและ $F$ เป็นฟังก์ชันนัลที่ขึ้นอยู่กับ $x=(x_1,x_2,...,x_d)^\top$, $u: \mathbb{R}^{c} \rightarrow \mathbb{R}$, $\nabla u (x) = ( \frac{\partial}{\partial x_1},\frac{\partial}{\partial x_2}, ... , \frac{\partial}{\partial x_d} )^\top$ สมมติให้ $\mathcal{J}$ หาอนุพันธ์แบบกาโตวได้ ดังนั้นเราจึงสามารถพิจารณาได้ว่า $F$ เป็นอนุพันธ์ย่อยที่ต่อเนื่องภายใต้ตัวแปรของมัน

เราจะใช้
\begin{align}
    \nabla_{u} F = \partial F / \partial u = F_u
\end{align}

แทนเกรเดียนต์ของ $F$ ภายใต้ $u$ เพื่อสร้างความแตกต่างกับเกรเดียนต์ของ $F$ ภายใต้ $x$ ซึ่งแืนสัญลักษณ์ของ
\begin{align}
    \nabla F = (\partial F / \partial x_1, ... \partial F / \partial x_d, )^\top
\end{align}

ในทำนองเดียวกันสัญลักษณ์เกรเดียนต์ของ $F$ ภายใต้ $\nabla u$ หมายถึง
\begin{align}
    \nabla_{\nabla u} F = (\partial F / \partial u_{x_1},..., \partial F / \partial u_{x_d})^\top \in \mathbb{R}^d
\end{align}

\hspace{1cm}ต่อไปเราจะพิจราณาปริภูมิของฟังก์ชันที่มีเงื่อนไขของขอบต่อไปนี้
\begin{align}
    \tilde{\mathcal{U}} = \{u \in \mathcal{U} | u = c \text{ บน } \partial \Omega \}
\end{align}

พร้อมกับปริภูมิของฟังก์ชันทดสอบกำหนดโดย
\begin{align}
    \tilde{\mathcal{V}} = \{v \in \mathcal{V} | v = 0 \text{ บน } \partial \Omega \}
\end{align}

\begin{Lemma}
    (จุดคงตัวของ $\mathcal{J}$) ฟังก์ชัน $u \in \mathcal{U}$ เป็นจุดคงตัวของฟังก์ชันนัลทั่วไป $\mathcal{J}$ (\ref{equation:general_functional}) ถ้าเงื่อนไข
    \begin{align}
        \int_{\Omega} \Big \langle \nabla_u F - \nabla \cdot \nabla_{\nabla u} F,v \Big \rangle_{\mathbb{R}^{d}} dx = 0
        \label{equation:lemma_stationarypoint}
    \end{align}
    \label{lemma:stationary_point}
    เป็นจริงสำหรับทุกๆ ฟังก์ชันทดสอบ $v \in \mathcal{V}$
\end{Lemma}

เห็นได้ชัดว่า (\ref{equation:lemma_stationarypoint}) เป็นจริงสำหรับฟังก์ชันใดๆ เมื่อ $\nabla_u F - \nabla \cdot \nabla_{\nabla u} = 0$ ดังนั้น $u \in \hat{u}$ เป็นจุดคงตัวของฟังก์ชันนัล $\mathcal{J}$ (\ref{equation:general_functional}) เมื่อ 
\begin{align}
    \nabla_{u} F - \nabla \cdot \nabla_{\nabla u} F = 0 \text{ บน } \Omega
    \label{equation:lemma_stationarypoint2}
\end{align}

โดยการใช้ผลของทฤษฏีบท \ref{theroem:nessery_minimizer} กับ (\ref{equation:lemma_stationarypoint2}) เป็นเงื่อนไขที่จำเป็นสำหรับโลคอลมินิไมเซอร์ของ (\ref{equation:general-minimize}) ซึ่งถ้า $ d > 1$ จะได้ว่า (\ref{equation:lemma_stationarypoint2}) จะนำไปสู่สมการเชิงอนุพันธ์ย่อย (ซึ่งรู้จักกันในชื่อของสมการออยเลอร์-ลากรางซ์) ที่มีเงื่อนไขค่าขอบ โดยจะเรียก (\ref{equation:general-minimize}) ว่า รูปแบบการแปรผัน (variational formulation) ของปัญหาค่าขอบ (\ref{equation:lemma_stationarypoint2}
) และถ้าเงื่อนไขค่าขอบนั้นถูกกำหนดไว้อย่างชัดเจนจะเรียกเงื่อนไขนี้ว่า เงื่อนไขจำเป็น (essential condition) และในทางกลับกันหากค่าขอบไม่ถูกกำหนดไว้อย่างชัดเจนจะเรียกว่า เงื่อนไขธรรมชาติ (natural condition) 

โดยสรุปทุกคำตอบ $u \in \mathcal{U}$ ในปัญหาค่าเหมาะสมที่สุด ดังเช่น (\ref{equation:general-minimize}) $\mathcal{J}$ ใน (\ref{equation:general_functional}) เป็นคำตอบของปัญหาค่าขอบที่ประกอบด้วยสมการ
\begin{align*}
    \nabla_u F - \nabla \cdot \nabla_{\nabla u} F = 0 \text{ บน } \Omega 
\end{align*}
และเงื่อนไขบนขอบธรรมชาติบน $\partial \Omega$

\begin{Example}
    ให้ $d = 2, \Omega = [0,1]^2, F = |\nabla u| \text{เมื่อ} u = u(\mathbf{x})$ จะได้ว่ารูปแบบเชิงแปรผันต่อไปนี้ 
    \begin{align*}
        \underset{u}{\min} \int_{\Omega} |\nabla u| d \Omega 
    \end{align*}
    เทียบเท่ากับปัญหาค่าขอบที่กำหนดโดย
    \begin{align}
        \left \{ \begin{array}{ll}  - \nabla \cdot  \Big( \dfrac{\nabla u}{|\nabla u|} \Big) = 0,  & \hspace{1cm} \mathbf{x} \in (1,n)^2 \\ \dfrac{\partial u}{\partial \boldsymbol{n}} = 0, & \hspace{1cm} \mathbf{x} \in \partial \Omega \end{array} \right .
    \end{align}
    กำหนดให้ \[
        \mathcal{R}(u) = \int_\Omega |\nabla u| d \Omega = \int_\Omega \sqrt{u_x^2+u_y^2} d \Omega
    \]
    เพื่อสะดวกในการคำนวณการแปรผันอันดับหนึ่งของ  $\mathcal{R}$ กำหนดให้ $\Phi (s)=s$ จะได้
    \begin{align*}
    \frac{\delta}{\delta u}\mathcal{R}(u;v) &= \left.\frac{d}{d\varepsilon}\mathcal{R}(u+\varepsilon v) \right|_{\varepsilon = 0} 
    = \left.\frac{d}{d\varepsilon} \int_{\Omega} \Phi (|\nabla (u + \varepsilon v)|)d\Omega 
    \right|_{\varepsilon = 0} 
    \end{align*}
    ดังนั้น
    \begin{align*}
    \frac{\delta}{\delta u}\mathcal{R}(u;v) 
    &= \int_{\Omega} \left.\frac{d}{d\varepsilon} \Phi ({\sqrt{(u_x+\varepsilon v_x)^2 + (u_y+\varepsilon v_y)^2 }})\right|_{\varepsilon = 0} d\Omega \\
    &= \int_{\Omega}^{}\left[\Phi'({\sqrt{(u_x+\varepsilon v_x)^2 + (u_y+\varepsilon v_y)^2 }}) \frac{(u_x+\varepsilon v_x)v_x}{{\sqrt{(u_x+\varepsilon v_x)^2+ (u_y+\varepsilon v_y)^2 }}}\right.\\
    &\quad + \Phi'({\sqrt{(u_x+\varepsilon v_x)^2 + (u_y+\varepsilon v_y)^2 }}) \left.\left.\frac{(u_y+\varepsilon v_y)v_y}{{\sqrt{(u_x+\varepsilon v_x)^2 + (u_y+\varepsilon v_y)^2 }}}\right] \right|_{\varepsilon = 0} d\Omega 
    \\
    &= \int_{\Omega} \Phi'(|\nabla u|)\left(\frac{u_x v_x}{|\nabla u|} + \frac{u_y v_y}{|\nabla u|} \right) d\Omega \\
    &= \int_{\Omega} \frac{\Phi'(|\nabla u|)}{|\nabla u|} (\nabla u \cdot \nabla v)d\Omega \\
    &= \int_{\Omega}\Phi'(|\nabla u|) \frac{\nabla u}{|\nabla u|} \cdot \nabla v d\Omega
    \end{align*}
    โดยเอกลักษณ์อันดับหนึ่งของกรีน จะได้ว่า
    \begin{align*}
    \frac{\delta}{\delta u}\mathcal{R}(u;v) &= - \int_{\Omega}^{} v \nabla \cdot \left(\Phi'(|\nabla u|) \frac{\nabla u}{|\nabla u|} \right)d\Omega + \int_{\partial \Omega}^{} v \left(\Phi'(|\nabla u|) \frac{\nabla u}{|\nabla u|} \cdot \boldsymbol{n}\right) dS \\
    &= -\int_{\Omega}^{} v \nabla \cdot \left(\Phi'(|\nabla u|) \frac{\nabla u}{|\nabla u|} \right)d\Omega + \int_{\partial \Omega}^{} v \left(\Phi'(|\nabla u|)  \frac{\partial u}{\partial \boldsymbol{n}}\right) dS
    \end{align*}
    เพราะฉะนั้น  
    \[
    \frac{\delta}{\delta u}\mathcal{R}(u;v) = -\int_{\Omega}^{} v \nabla \cdot \left(\Phi'(|\nabla u|) \frac{\nabla u}{|\nabla u|} \right)d\Omega + \int_{\partial \Omega}^{} v \left(\Phi'(|\nabla u|)  \frac{\partial u}{\partial \boldsymbol{n}}\right) dS
    \]
    เมื่อ $  \boldsymbol{n} $ แทนเวกเตอร์หนึ่งหน่วยที่ตั้งฉากกับขอบของภาพในทิศทางชี้ออก

    จะได้สมการออยเลอร์ที่สมนัยกับปัญหานี้คือ
    \begin{align*}
        - \Delta u = 0 \text{ บน } \Omega \\
        \frac{ \partial  u }{ \partial \boldsymbol{n}} = 0 \text{ บน } \partial \Omega
    \end{align*}
\end{Example}