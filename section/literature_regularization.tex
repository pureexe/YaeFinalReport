\section{วิธีการเร็กกิวลาร์ไลซ์เซชัน}

\hspace{1cm} จากโครงงานวิจัยนี้จะใช้ตัวแบบ ROF ที่คิดค้นโดยคุณ Chan และคุณ Shen \cite{ref:rof-inpaint-chan-shen} ซึ่งจะมีตัวแบบดังนี้

\begin{align*}
    \underset{u}{{min}} \{ \mathcal{J}(u) = \lambda \mathcal{D}(u) + \mathcal{R}(u) \}
\end{align*}

\hspace{1cm} โดยจะพจน์ $\mathcal{D}$  ว่าพจน์วัดค่าเหมาะสมข้อมูล และ $\mathcal{R}$ ว่าพจน์เร็กกิวลาร์ไลซ์เซชัน ซึ่งพจน์วัดค่าเหมาะสมนี้จะคอยปรับให้ภาพผลลัพธ์ให้ภาพที่ออกมามีค่าใกล้เคียงกับผลลัพธ์ ส่วนพจน์เร็กกิวลาร์ไลซ์เซชันจะคอยทำให้ภาพแตกต่างไปจากของเดิม โดยมี $\lambda$ คอยเป็นพารามิเตอร์ที่กำหนดผลกระทบของพจน์เร็กกิวลาร์ไลซ์เซชัน

\hspace{1cm} หาก $\lambda$ เป็น 0 จะมีแต่พจน์เร็กกิวลาร์ไลซ์เพียงเท่านั้น ส่งผลให้ภาพที่ได้ไม่มีเค้าโครงของภาพเดิมเหลืออยู่เลย แต่หากกำหนดค่า $\lambda$  มีค่ามากเกินไปจะทำให้พจน์วัดค่าเหมาะสมส่งผลมาก ซึ่งทำให้ภาพผลลัพธ์ที่ได้มาเหมือนของเดิมจนแทบไม่เกิดความแตกต่างของการต่อเติมภาพ