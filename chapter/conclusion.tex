\chapter{สรุป}

\hspace{1cm}สำหรับโครงงานวิจัยนี้ได้ทำการพัฒนาขั้นตอนวิธีเชิงตัวเลขสำหรับการซ่อมแซมภาพศิลปะไทยและการลบบทบรรยายอนิเมะได้มีผลการดำเนินงานทั้งสิ้นดังนี้

\hspace{1cm} ขั้นตอนวิธีเชิงตัวเลขทั้ง 3  วิธีสำหรับแก้ตัวแบบการต่อเติมภาพด้วยการแปรผันรวม ได้แก่วิธีเดินเวลาแบบชัดแจ้ง (Algorithm \ref{algorithm:explicit_timemarch}) วิธีทำซ้ำจุดตรึง (Algorithm \ref{algorithm:fixedpoint}) และวิธีการสปริทเบรกแมน (Algorithm \ref{algorithm:splitbreg,am}) พบว่า วิธีการสปริทเบรกแมนมีคุณภาพที่ดีกว่าเมื่อวัดด้วยค่า PSNR และ SSIM ส่วนเวลาท่ใช้ประมวลผลพบว่าวิธีสปริทเบรกแมนใช้เวลาน้อยกว่าวิธีเดินเวลา 8 เท่าและใช้เวลาน้อยกว่าวิธีทำซ้ำจุดตรึง 5 เท่า

\hspace{1cm} พีระมิดรูปภาพใช้เพื่อเพิ่มความเร็วในการประมวลผลของวิธีการสปริทเบรกแมน พบว่าการทำซ้ำบนพีระมิดรูปภาพด้วยการทำซ้ำ 10/3/3/10 ใช้เวลาประมวลผลน้อยที่สุดและให้ค่าคุณภาพทั้งในด้าน PSNR และ SSIM ใกล้เคียงกวิธีอื่นจึงเลือกใช้พีระมิดรูปภาพนี้ในการซ่อมแซมรูปภาพ

\hspace{1cm} การซ่อมแซมภาพศิลปะไทย การใช้วิธีการสปริทเบรกแมนพร้อมทั้งการใช้พีระมิดรูปภาพ (Algorithm \ref{algorithm:MultiSplitBregmanColorInpaint}) พบว่าภาพที่ผ่านการต่อเติมให้คุณภาพดีกว่าเมื่อวัดด้วย PSNR และใช้เวลาน้อยกว่าประมาณ 7 เท่า

\hspace{1cm} ขั้นตอนวิธีค้นหาคำบรรยายในภาพด้วยวิธีการที่คิดค้นขึ้น (Algorithm \ref{algorithm:FindSubtitle}) พบว่ามีความผิดพลาดในการตรวจหาพิกเซลที่เป็นบทบรรยายอยู่ที่ร้อยละ 11.98 ซึ่งจะใช้ขั้นตอนวิธีนี้ในการหาคำบรรยายเพื่อทำการลบถัดไป

\hspace{1cm} เนื่องจากวิดีโอเป็นชุดของภาพคณะผู้วิจัยจึงได้เสนอขั้นตอนวิธีข้ามเฟรม (Algorithm \ref{algorithm:subtitle_borrowframe})  ขั้นตอนวิธียืมเฟรม (Algorithm \ref{algorithm:subtitle_skipframe}) และขั้นตอนวิธียืมเฟรมและข้ามเฟรม (Algorithm \ref{algorithm:subtitle_skipnborrowframe}) เพื่อช่วยในการลดเวลาการประมวผล พบว่าวิธียืมเฟรมและข้ามเฟรมใช้เวลาประมวลผลน้อยสุด ซึ่งใช้เวลาน้อยกว่าวิธีสปริทเบรกแมนและพีระมิดรูปภาพบนวิดีโอถึง 2 เท่า

\hspace{1cm} สำหรับการลบบทบรรยายอนิเมะเมื่อใช้วิธีตรวจหาคำบรรยายที่คิดข้นขึ้นพร้อมทั้งใช้วิธีการเดียวกับที่ใช้สำหรับซ่อมแซมศิลปะไทย รวมกับการวิธีการขยืมเฟรมและข้ามเฟรม (Algorithm \ref{algorithm:subtitle_skipnborrowframe}) พบว่าใช้เวลาน้อยกว่าวิธีการสปริทเบรกแมน (Algorithm \ref{algorithm:splitbregman}) บนวิดีโอ 67 เท่า

\clearpage