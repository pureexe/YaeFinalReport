\begin{algorithm}[H]
    \SetAlgoNoLine
    \caption{วิธีสปริทเบรกแมนสำหรับการต่อเติมภาพที่ใช้การแปรผันรวม}
    \SetKwFunction{FMain}{$u \longleftarrow SplitBregman$}
    \SetKwProg{Fn}{}{}{}
    \KwIn{
		\\
		\hspace{1cm} $u$ คือรูปภาพที่ต้องการต่อเติม \\
		\hspace{1cm} $\lambda$ คือพารามิเตอร์เร็กกิวลาร์ไรเซชัน ที่ได้กล่างถึงในสมการ (\ref{e2}) \\
        \hspace{1cm} $\theta$ คือพารามิเตอร์เพนัลที ซึ่งเป็นจำนวนจริงบวก\\
        \hspace{1cm} $N_{gs}$ เป็นจำนวนเต็มบวกสำหรับกำหนดจำนวนรอบที่ทำงานของการทำเกาส์-ไซเดล \\
		\hspace{1cm} $N$ เป็นจำนวนเต็มบวกสำหรับกำหนดจำนวนรอบที่ทำงานของสปริทเบรกแมน \\
		\hspace{1cm} $\varepsilon$ เป็นจำนวนจริงบวกของค่าความคลาดเคลื่อนสัมพัทธ์ \\
	}
	\KwOut{รูปภาพที่ผ่านการต่อเติมแล้ว}	
    \Fn{\FMain{$u,\lambda, \theta, N_{gs}, N, \varepsilon$}}{
        \textbf{initialize}
        $i = 0$,
        $\boldsymbol{b} = \vec{0}$,
        $\boldsymbol{w} = \vec{0}$,
        $z = u$ \\
        \While{$ i < N $ \textbf{and} $err > \varepsilon$}{
            $u^{old} = u$; 
            $w^{old} = w$;
            $b^{old} = b$;
             \\
            $u^{new}=\underset{u}{\arg\min} \{ \mathcal{J}_1(u) = \dfrac{1}{2} \int_{\Omega} \lambda(u-z)^2 d\Omega + \frac{\theta}{2} \int_{\Omega} (\boldsymbol{w}^{\text{old}} - \nabla u + \boldsymbol{b}^{\text{old}}) d\Omega \}$ \\
            $w^{new} =\underset{\boldsymbol{w}}{\arg\min} \{ \mathcal{J}_2(\boldsymbol{w}) = \int_{\Omega}  | \boldsymbol{w}|  d\Omega  + \frac{\theta}{2} \int_{\Omega} (\boldsymbol{w} - \nabla u^{\text{New}} + \boldsymbol{b}^{\text{old}}) d\Omega \}$\\
            $ b^{new} = b^{old} + \nabla u^{new} - \boldsymbol{w}^{new} $ \\
            $err = \frac{||u^{new}-u^{old}||}{||u^{new}||}$ \\
            $ i = i + 1 $ \\
        }
    }
\end{algorithm}