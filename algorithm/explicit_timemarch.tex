\begin{algorithm}[H]
    \caption{วิธีการเดินเวลาแบบชัดแจ้งสำหรับการต่อเติมภาพที่ใช้การแปรผันรวม}
    \KwIn{
		\\
		\hspace{1cm} $u$ คือรูปภาพที่ต้องการต่อเติม \\
		\hspace{1cm} $\lambda$ คือพารามิเตอร์เร็กกิวลาร์ไรเซชัน ที่ได้กล่างถึงในสมการ (\ref{e2}) \\
		\hspace{1cm} $\beta$ เป็นจำนวนจริงบวกที่ใช้เพื่อหลีกเลี่ยงการหารด้วยศูนย์\\
		\hspace{1cm} $\tau$ เป็นจำนวนจริงบวกที่เป็นตัวแปรเดินเวลา \\
		\hspace{1cm} $N$ เป็นจำนวนเต็มบวกสำหรับกำหนดจำนวนรอบที่ทำงาน \\
		\hspace{1cm} $\varepsilon$ เป็นจำนวนจริงบวกของค่าความคลาดเคลื่อนสัมพัทธ์ \\
	}
	\KwOut{รูปภาพที่ผ่านการต่อเติมแล้ว}
    \SetKwFunction{FMain}{$u \longleftarrow ExplicitTimeMarching$}
    \SetAlgoNoLine 
    \SetKwProg{Fn}{}{}{}
    \Fn{\FMain{$u,\lambda,\beta,\tau,N,\varepsilon$}}{
        \textbf{initialize}
        $i = 0$; $z = u$; $err = 1$\\
        \While{$ i < N $ \textbf{and} $err > \varepsilon$}{
            $u^{old} = u$\\
            $u = u + \tau\left(\nabla \cdot\left(\dfrac{\nabla u}{\sqrt{u_x^2 + u_y^2+ \beta}}\right) + \lambda(u-z) \right)$ \\ 		
            $err = \frac{||u-u^{old}||}{||u||}$ \\
            $ i = i + 1 $
        }
    }
\end{algorithm}