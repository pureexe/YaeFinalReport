\begin{algorithm}[H]
    \label{algorithm:MultiSplitBregmanColorInpaint}
    \caption{วิธีสปริทเบรกแมนที่ใช้พีระมิดรูปภาพ}
    \KwIn{
        \\
        \hspace{1cm} $u$ คือรูปภาพที่ต้องการต่อเติม \\
		\hspace{1cm} $\lambda$ คือพารามิเตอร์เร็กกิวลาร์ไรเซชัน ที่ได้กล่างถึงในสมการ (\ref{e2}) \\
        \hspace{1cm} $\theta$ คือพารามิเตอร์เพนัลที ซึ่งเป็นจำนวนจริงบวก\\
        \hspace{1cm} $N_{gs}$ เป็นจำนวนเต็มบวกสำหรับกำหนดจำนวนรอบที่ทำงานของการทำเกาส์-ไซเดล \\
        \hspace{1cm} $c$ ตัวแปรช่วยสำหรับบอกความลึก ให้กำหนดเป็น 1 \\
        \hspace{1cm} $m$ คือ ระดับความลึกของพีระมิดรูปภาพ เป็นจำนวนเต็มบวก \\
        \hspace{1cm} $N_0$ จำนวนรอบการทำสปริทเบรกแมนที่ชั้นละเอียดสุด \\
        \hspace{1cm} $N_1$ จำนวนรอบการทำสปริทเบรกแมนที่ชั้นต่างๆ \\
        \hspace{1cm} $N_2$ ตำนวนรอบการทำสปริทเบรกแมนที่ชั้นหยาบสุด \\
	}
	\KwOut{รูปภาพที่ผ่านการต่อเติมแล้ว}
    \SetAlgoNoLine
    \SetKwFunction{FMain}{$u \longleftarrow MultiSplitBregmanColor $}
    \SetKwProg{Fn}{}{}{}
    \Fn{\FMain{$\boldsymbol{u}, \lambda, \theta, N_{gs}, N_0,N_1,N_2, \varepsilon,c,m$}}{
        \textbf{Initialize} $height = $ ความสูงของภาพ $\boldsymbol{u}$, $width = $ ความกว้างของภาพ $\boldsymbol{u}$ \\
        \If{c < m}{
            $\boldsymbol{x} = Bilinear(\boldsymbol{u},\lfloor width * 0.5 \rfloor,\lfloor height * 0.5 \rfloor)$\\
            $y = Bilinear(\lambda,\lfloor width * 0.5 \rfloor,\lfloor height * 0.5 \rfloor)$\\
            $r = MRSBC(\boldsymbol{x},\boldsymbol{z},y, \lambda, \theta,$ \\$ \hspace{1cm}  N_{gs}, N_0, N_1, N_2, \varepsilon,c+1,m)$\\
            $\boldsymbol{u} = Bilinear(r,width,height)$\\
        }
        \uIf{$c = 1$}{$N_{SB}=N_0$}
        \uElseIf{$c = m$}{$N_{SB} = N_2$}
        \Else{
            $N_{SB} = N_1$
        }
        $u = SplitBregmanColor(\boldsymbol{u}, \lambda, \theta, N_{gs}, N_{SB}, \varepsilon) $  \\
    }
\end{algorithm}